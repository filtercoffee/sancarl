\documentstyle{article}
\oddsidemargin=.2in
\evensidemargin=.2in
\textwidth=6in
\topmargin=-.5in
\textheight=9in
\parindent=0in
\pagestyle{empty}

\begin{document}
\section*{Chapter 5}
Model Government as entity that consumes G of output but must abide by Government Budget Constraint (no budget surpluses or deficits)

\begin{eqnarray*}
G &=& T \Longrightarrow \mbox{ Government spending ... taxes }\\
\mbox{Exogenous Variables } &=& (G, Z, K)\\
\mbox{Endogenous Variables } &=& (C, N^s, N^d, T, Y, w)\\
\mbox{Competitive Equilibrium } &\Longrightarrow & \mbox{ consumes and forms arc price (i.e., wage) takers.}\\
&\Longrightarrow & \mbox{ equilibrium wage is above } N^s(w) = N^D(w)\\
&\Longrightarrow & \mbox{ Agent Max } U[c,l ] \mbox{ s.t., budget constant } C = D-T+wN^s\\
&\Longrightarrow & \mbox{ firms maximizes profits } D = \max_{N} \left\{ ZF(K, N^D) - wN^D \right\}\\
G &=& T
\end{eqnarray*}

Note: Combining \begin{list}{*}{}
\item $C = D - T + w N^s$
\item $D = Y - wN$
\item $G = T$
\end{list} 

We find $Y = C + G = $ Income/Expenditure equality\\

Define : Products Possibility Factor (PPF) = Consumption as a factor of leisure\\
$C = Y - G = ZF(K, h-l) - G$\\
$C(l) = ZF(K, h-l) - G$ taking $(Z, K, h, G)$ as exogenous.\\

Define : Marginal Rate of Transformation (MRT) = rate can convert one good into another\\
\begin{eqnarray*}
-\frac{\partial c}{\partial l} &=& -\frac{\partial Y}{\partial N}\frac{\partial N}{\partial l} = \frac{\partial Y}{\partial N} \Rightarrow\mbox{  since, } \frac{\partial N}{\partial l} = -1\\
MRT_{l,c} = MP_N &=& -\mbox{slope of PPF}\\
\end{eqnarray*}

Next step : combine PPF with indifference curves.\\

At equilibrium, Tangent of PPF = tangent of indifference curves.\\

-slope at the tangent will equal minus wage\\

$\Longrightarrow MRS_{l,c} = MRT_{l,k} = MP_N$\\

Proof : \\
\begin{eqnarray*}
D &=& \max_{N} \left[ Y(N) - wN\right]\\
\mbox{FOC : } 0 &=& \frac{\partial Y}{\partial N} - w\\
\mbox{Consumer : } \max U\left[ c, l=h-N \right] s.t., C &=& D - T + w N\\
&=& U\left[ c, l=h-N \right] + \lambda \left[ D - T + wN - C\right]\\
\mbox{FOC : } \frac{\partial}{\partial N} : 0 &=& U_l(-1) + \lambda w\\
\frac{\partial}{\partial c} : 0 &=& U_c - \lambda \\
\frac{U_l}{U_c} &=& w
\end{eqnarray*}

\subsection*{Optimality}
Connection betwen competitive ... and economic efficiency ... for two resons.\\
\begin{list}{-}{}
\item Show how free markets can produce socially optional outcomes.
\item ... to analyze social optimum then a competitive equliibrium.
\end{list}

Define : A competitive equlibrium is "Pareto Optional" if there is no way to make ... better off without ... wage off.\\

Introduce social planner $ \Rightarrow $ no need for markets, and thus no need for prices/wages.\\
\begin{eqnarray}
\mbox{Objective : } \max_{(c,l)} U(c,l) \mbox{ such that, } C &=& D-T +w(h-l)\nonumber\\
&=& Y(h-l)-T\nonumber\\
...&=&U(c,l) + \lambda (Y(h-l) - T - C)\nonumber\\
\frac{\partial}{\partial c} : 0 &=& U_c - \lambda\nonumber\\
\frac{\partial}{\partial l} : 0 &=& U_l - \lambda\frac{\partial Y}{\partial N} (-1)\nonumber\\
U_c &=& \lambda\nonumber\\
U_l &=& \lambda \frac{1}{N}\nonumber\\
\frac{U_l}{U_c} &=& \frac{1}{N}
\end{eqnarray}

Thus, MPL = MRS, just as in the competitive equlibrium case.\\

First Fundamental Theorem of Welfare Economics : under certain conditions, competitive equlibirum is Pareto Optimal.\\

Second Fundamental Theorem of Welfare Economics : under certain conditions, a Pareto Optimal is competitive equlibirum.\\

Adam Smith, wealth of nations, `` Unrestricted market economy would behave as it ... invisible hand were ... actions that were beneficial for all''\\

\subsection*{Sources of Social Inefficiencies}
\begin{list}{}{}
\item (1) Externalities : ... problem is that there is not a market to ... externaly\\
negative: firm does not pay for pollution\\
positive: frim not compensated for external benefit\\
\item (2) Distorting taxes : For example, proportional wage tax\\
effective wage $\equiv w(1-T)$\\
$\left. 
\begin{array}{ll}
\mbox{Consumer optimization : $MRS = w(1-T)$}\\
\mbox{Firm optimization : $MPL = w$}
\end{array} 
\right\}\mbox{$MRS < MPL$} $ \\

Tax derives a wedge between MRS and MPL\\

\item (3) Firms (or workers) not price takers.\\
\item (4) Incomplete Markets (Externality is a special case)\\
\end{list}

$\Delta G\uparrow $ on : \\
\begin{list}{}{}
\item (1) $\Delta Y \uparrow$
\item (2) $\Delta c \downarrow$
\item (3) $\Delta l \downarrow$
\item (4) $\Delta w \downarrow$
\end{list}

Basically, increase in ... spending, and hence, an increase in ... syntax, has a negative increase effect. This ... both consumption and leisure. Less leisure means more labor, means more output. More labor supply imples lower wages.\\

$\Delta Z\uparrow $ on : \\
\begin{list}{}{}
\item (1) $\Delta Y \uparrow$
\item (2) $\Delta c \uparrow$
\item (3) $\Delta l $ intermediate; need it to drop for ... with ... cycles.
\item (4) $\Delta w \uparrow$
\end{list}

$Z \uparrow$ imples $MPL = \frac{\partial V}{\partial N}$ increases. This will create a positive income effect, but it will also make leisure more expensive (i.e., wages are higher )thus there is a susbititution effect causing consumption further, bit more than leisure indeterminate.\\

For model to be consistent with business cycling need $\Delta Z \uparrow \Rightarrow \Delta N \uparrow (\Delta l \downarrow)$  tat is, for substitution effect to dominate income effect.  Now, in long run, since $N = constant$, we have income effect ... substitution effect. But, can argue that ..., substitution effect dominates. \\

Emprically, shocks to productivity appear to play a role in business cycles.\\

Diluting tax on Wage Income\\
\end{document}