\documentstyle{article}
\oddsidemargin=.2in
\evensidemargin=.2in
\textwidth=6in
\topmargin=-.5in
\textheight=9in
\parindent=0in
\pagestyle{empty}

\begin{document}
\section*{Chapter 6}
Malthos ($\sim$ 1800) : Any technological improvement leads to increased population growth, so that in the long run there is no improvement in the standard of living. $\Rightarrow$ Only means for improving standard of living is population control.\\

Sotone (1950) : Only technological improvements can improve GDP per agent in the long run. In the short run, standard of living can improve if agents save more. $\Rightarrow$ Technological growth exogenous, in contrast to ``Endogenous growth model''\\

Emprical Facts:\\
\begin{list}{ }{}
\item (1) Before industrial revolution ($\sim$ 1980), standards of living differed little across tome and across countries.
\item (2) Since Industrial Revolution, per capita income growth has been sustained in the wealthiest countries.
\item (3) Positive correlation between investment and output.
\item (4) Negative correlation between population growth and output per capita.
\item (5) Income growth dispersion widened between 1800-1950.
\item (6) No correlation between level and growth rate $\Rightarrow$ No convergence.
\item (7) High level countries have similar growth, low level countries have diffuse growth.
\end{list}

Malthus : Replace capital K with Land L (which is in fixed supply)\\
Persihable output follows:
\begin{eqnarray*}
Y &=& ZF(L,n)
\end{eqnarray*}

There is no investment, no storage, no technology to convert food to capital. Each agent has one unit of labor to supply, no utility for leisure. $\Rightarrow$ N is both population and labor input.\\

Assume population growth depends on consumption per worker.\\
\begin{eqnarray*}
\frac{N(*+1)}{N(*)} = \frac{N^\prime}{N} &=& g \left[\frac{c}{N}\right] = g\left[\frac{ZF(L,N)}{N}\right]\\
\mbox{In equilibrium, all goods are consumed : } C = Y &=& Z F(L,N)\\
\mbox{From constant returns to scale : } \frac{ZF(L,N)}{N} &=& ZF \left(\frac{L}{N}, 1\right)\\
\Rightarrow N(* + 1 ) &=& g\left[ ZF \left(\frac{L}{N}, 1\right)\right]\cdot N(*)
\end{eqnarray*}
Assuming RHS is increasing, concave in N, with initial slope greater than one\\$\Rightarrow$ as such, there exists a steady state $N^*$ where $N(*+1) = N(*)$, or $N^*$ is defined via : 
\begin{eqnarray*}
1 &=& g\left[ ZF\left(\frac{L}{N^*}, 1\right)\right]
\end{eqnarray*}

Analysis of steady state in Malthusian Model :\\
\begin{eqnarray*}
\mbox{Constant returns to scale : } \frac{Y}{N} &=& ZF(\frac{L}{N}, 1)\\
\mbox{Define output per worker : } \frac{Y}{N} \equiv Y &=& \frac{c}{N} = \mbox{consumption per worker}\\
\mbox{Define Land per worker : } \frac{L}{N} &=& l\\
\Rightarrow c &=& Y = ZF(l)\\
\Rightarrow N(*+1) &=& g(c).N(*)
\end{eqnarray*}

Steady State : $(c^*, l^*)$ and thus $N^* = \frac{L}{l^*}$, determined via\\
\begin{eqnarray*}
g(c^*) &=& 1\\
c^* = g^{-1} [1] &=& ZF(l^*)
\end{eqnarray*}

Interpret : Long-run standard of living, $c^* = $ consumption per capita, uniquely determined by the function $g(\cdot)$.\\

Note : This is completely independent of the value of Z : if Z were to increase once, this would ultimately lead only to an increase in population.\\

Note : Although Williams does not say so, a consant increase in Z would imply that a steady state would not exist, and might imply that $\frac{Y}{N} = \frac{C}{N} = ZF\left( \frac{L}{N}, 1\right)$ might increase; I guess it depends how fast Z increases versus $\left( \frac{L}{N}, 1\right)$ decreases. Actually, since each steady state ahs $c^* = g^{-1} [1]$, I am guessing a constantly increasing Z does little to $c^*$. Probably it causes it above $c^*$, but does not increase out time. (unless Z accelerates).\\

Malthus model does not alter 1800 due to \\
\begin{list}{ }{}
\item (1) Captial, which can increase, replacing fixed land as input to production
\item (2) Agents choosing to have less children.
\end{list}

Solow Model:\\
Exogenous population growth (in contrast to Malthusian Model)
\begin{center}
$N(*+1) = (1+n) N(*)$\\
\end{center}

Consumers do not value leisure, so offer their one unit of labor at market price thus population is also the labor force.\\

No taxes. Income generated from labor and dividends. Agnets choose how much to consume, and how much to save. (In contrast to Malthus.) Assume a fraction ``S'' of output is saved.\\
\begin{eqnarray*}
C &=& (1-s)Y\\
S &=& s Y\\
\mbox{Output has constant returns to scale:}\\
\frac{Y}{N} &=& ZF\left( \frac{K}{N}, 1\right)\\
Y &=& ZF(K), where, Y = \frac{Y}{N}, K = \frac{K}{N}\\
\end{eqnarray*}
Marginal product of capital, $MPK = F_k$, is positive, but $F_{kk}<0$ \\

Depreciation : \\
$K(*+1) = (1-\delta) K(*) + I$, where I = investment\\

Competetive Equlibrium:
\begin{list}{ }{}
\item (1) Consumption traded for labor at price/wage = w
\item (2) Consumption traded for capital at price = 1
\end{list}
\begin{eqnarray*}
\mbox{Labor Market clears with supply :} N &=& Demand(w)\\
\mbox{Capital Market clears with } S &=& I\\
K(*+1) = (1-\delta) K(*) + SZNF\left( \frac{K(*)}{N(*)}\right)\\
\frac{K(*+1)}{N(*+1)} \frac{N(*+1)}{N(*)} &=& (1-\delta) \frac{K(*)}{N(*)} + SZNF\left( \frac{K(*)}{N(*)}\right)\\
K(*+1) (1+n) &=& SZF(K(*)) + (1-\delta) K(*)\\
K(*+1) &=& \frac{SZF(K(*))}{1+n} + \frac{(1-\delta) K(*)}{1+n}\\
\mbox{Note : } \frac{\partial}{\partial K}\left[ \frac{SZF(K(*))}{1+n} + \frac{(1-\delta) K(*)}{1+n}\right] &=& \frac{SZF^\prime(K)}{1+n} + \frac{1-\delta}{1+n} > 0\\
\frac{\partial}{\partial K}\left[ \frac{SZF(K(*))}{1+n} + \frac{(1-\delta) K(*)}{1+n}\right] &=& \frac{SZF^{\prime\prime}(K)}{1+n} < 0 
\end{eqnarray*}
By mapping this function with a $45\deg$ line, can determine evolution. There exists a steady state level of capital per agent, call it $K^*$ \\

It thus follows that output per workder $\equiv Y^* = ZF(K^*)$ and consumption per worker $\equiv C^* = (1-S)ZF(K^*)$\\

Thus, if Z is constant, $C^*$ cannot grow.\\

Competitive Statistics : How is $K^*$ affected by a change in 
\begin{list}{ }{}
\item (1) S
\item (2) n
\item (3) Z
\end{list}

Steady state $K^*$ defined implicitly by 
\begin{eqnarray*}
K^* &=& \frac{SZF(K^*)}{1+n} + \frac{(1-d)K^*}{1+n}\\
SZF(K^*) &=& (n+d) K^*
\end{eqnarray*}

Convenient to graph LHS and RHS as a function of $K^*$\\

Graphically, easy to see that $K^*$ increases with S.\\

Mathematically, we get
\begin{eqnarray*}
ZF+SZF^\prime K_S &=& (n+d)K_S\\
K_S &=& \frac{ZF}{n+d-SZF^\prime}\\
&=& \frac{ZF}{SZ\left[ \frac{F}{K} F_K\right]}
\end{eqnarray*}

Due to $F_{KK}<0$, the denominator is positive, and therefore so also is $K_S$\\

Note: This does not affect growth rate, as new equilibrium will still have a constant income per capita, and thus aggregate income grows at the population rate.\\

Golden Rule : While inccreasing savings increases $K^*$ and thus $Y^*$, the more important variable is $C^*$, consumption per wokrer. Clearly, if we set $S=1$, output would be maximized, but consumption wiould be zero. This imples a local optimum $C^*$, and thus an optimal $S^*$.
\begin{eqnarray*}
C^* &=& (1-S) ZF(K^*)\\
&=& ZF(K^*) - (n+d)K^*\\
Optimum : \frac{\partial C^*}{\partial K^*} &=& 0\\
&=& ZF^\prime (K) - (n+d)\\
&\Rightarrow & MPK = n+d
\end{eqnarray*}

Second Method : Thik of $K^*$ as a function of S\\
\begin{eqnarray*}
C(S) &=& (1-S) ZF(K(S))\\
Optimum : o &=& C_S\\
&=& (1-S) ZF_K K_S - ZF(K(S))
\end{eqnarray*}
where $K(S)$ defined via $SZF(K(S)) = (n+d) K(S) \forall S$\\

Differentiating w.r.t S:
\begin{eqnarray*}
ZF + SZF_K K_S &=& (n+d) K_S\\
K_S\left[ n+d-SZF_K \right] &=& ZF
\end{eqnarray*}
Plug back into $0 = C_S$ equation:\\
\begin{eqnarray*}
(1-S) ZF_K \left( \frac{ZF}{n+d-SZF_K}\right) &=& ZF\\
ZF_K(1-S) &=& n+d-SZF_K\\
MPK &=& ZF_K\\
&=& n+d
\end{eqnarray*}
Increase in Population Growth:\\
More workers implies more capital has to be saved. In equilibrium, both $K^*$ and $C^*$ will be lower.\\

\begin{eqnarray*}
SZF(K(n)) &=& (n+d) K(n)\\
\frac{\partial}{\partial n} : SZF_K K_n &=& K + (n+d) K_n\\
K_N\left[ SZF_K -n -d\right] &=& K\\
K_n &=& \frac{K}{SZF_K -n - d}\\
&=& \frac{K}{SZ\left[F_K - \frac{F}{K}\right]} < 0\\
C_n &=& (1-S) Z F_K K_n <0
\end{eqnarray*}
Increase in productivity : The Solow model predicts that standard of living can only increase in the long run if productivity increases in the long run.
\begin{eqnarray*}
SZF(K(Z)) &=& (n+d) K(Z)\\
\frac{\partial}{\partial Z} : SZF_K K_Z + SF &=& (n+d) K_Z\\
K_Z &=& \frac{SF}{SZ\left[\frac{F}{K} - F_K\right]} > 0\\
C_Z &=& \frac{\partial}{\partial Z} \left[ (1-S) ZF(K(Z)) \right]\\
&=& (1-S) \left\{ F + ZF_K K_Z\right\} > 0
\end{eqnarray*}
Growth Accounting : Determine what proportion of output is due to changes in (Z, K, N)\\

Assuming Cobb-Douglass, $Y(*) = Z(*) K(*)^{0.36} N(*)^{0.64}$ \\
Can measure $(Y, K, N)$ directly; thus, can back out $Z(*)$\\

Find a ``Productivity slow down'' from 1948 $\sim$ 1982; 3 possible reasons -
\begin{list}{ }{}
\item (1) Measurement Problem : Shift from manufacturing services.
\item (2) Increases in energy prices : Capital made obsolete not adequateley accounted for.
\item (3) Costs of adopting new technology.
\end{list}

\end{document}