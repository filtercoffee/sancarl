\documentclass[titlepage,11pt]{article}
\usepackage{graphicx}
\usepackage{latexsym}
\usepackage{amssymb}
%\usepackage{amsmath}
%\usepackage{times}
\usepackage{subfig}
\usepackage{epsfig} \setlength{\textheight}{8.5in}
\setlength{\textwidth}{6.2in}
\setlength{\oddsidemargin}{0.1in} \setlength{\evensidemargin}{0.1in}
\def\big{\begin{array}{c} \\ \end{array} \!\!\!\!\!}
\def\bbig{\begin{array}{c} \\ \\ \end{array} \!\!\!\!\!}
\def\sm{\begin{array}{c}  \end{array} \!\!\!\!\!}
\setlength{\topmargin}{-0.46in}
\def\bq{\begin{equation}}
\def\eq{\end{equation}}
\def\by{\begin{eqnarray}}
\def\ey{\end{eqnarray}}
\def\byy{\begin{eqnarray*}}
\def\dq{d{\bf 1}}
\def\eyy{\end{eqnarray*}}
\def\bdi{\begin{displaymath}}
\def\d{{\scriptstyle \Delta}}
\def\dd{{\scriptscriptstyle \Delta}}
\def\ds{{\scriptscriptstyle \triangle}}
\def\h3{\hspace*{2.mm}}
\def\edi{\end{displaymath}}
\newcommand{\refeq}[1]{(\ref{#1})} % produces "(1)" -- no need to type brackes

\newtheorem{proposition}{Proposition}
\newtheorem{proof}{Proof}
\newtheorem{lemma}{Lemma}
\newtheorem{corollary}{Corollary}

\def\nn{\nonumber}

%% all you have to do is comment out the first two and in the others and all should work!
%\def\pcd{gau/}
%\def\pcdbib{../../artfi2}
\def\pcd{}
\def\pcdbib{artfi5}

\begin{document}
\setlength{\baselineskip}{18pt}

\newcommand{\eproof}{{\hfill $\Box\,\,\,\,\,\,\,\,\,\,$}}
\newcommand{\supp}[1]{^{^{{#1}}}}
\newcommand{\M}{{\cal{M}}}
\newcommand{\Yb}{{\bf Y}}
\newcommand{\Q}{{\cal Q}}
\newcommand{\R}{{\cal R}}
\newcommand{\F}{{\cal F}}
\newcommand{\pr}{{\cal P}}
\newcommand{\Tb}{{\hat{T}}}
\newcommand{\ER}[1]{{\rm E^\R}\left[{#1}\right]}
%\newcommand{\E}[1]{{\rm E^\Q}\left[{#1}\right]}
\newcommand{\E}{{\rm E}}
\newcommand{\sub}[1]{_{_{{#1}}}}
\newcommand{\ind}[1]{{\bf 1}\sub{\left\{{#1}\right\}}}
\newcommand{\ignore}[1]{ }
\newcommand{\One}{{\bf 1}}
\newcommand{\hlf}{\frac{1}{2}}

%%ENRON SPECIFIC
\newcommand{\Ph}{p^H}
\newcommand{\Pl}{p^L}
\newcommand{\Qh}{q^H}
%\newcommand{\lb}{\overline{\lambda}}
\newcommand{\lH}{\lambda\supp{H}}
\newcommand{\lL}{\lambda\supp{L}}
%\newcommand{\lij}{\lambda\supp{j}\sub{i}}
\newcommand{\lij}{\lambda\sub{ij}}
%\newcommand{\lkj}{\lambda\supp{j}\sub{k}}
%\newcommand{\aij}{\alpha\supp{j}\sub{i}}
\newcommand{\aij}{\alpha\sub{ij}}
\newcommand{\pj}{p\supp{j}\!}
\newcommand{\pH}{p\supp{H}\!}
\newcommand{\tm}{t^{-}}
\newcommand{\qj}{q\supp{j}\!}
\newcommand{\bx}{\overline{\xi}}

%%CAMPCOCH
\newcommand{\C}{{\widehat{C}}}
\newcommand{\kr}{\kappa\sub{r}}
\newcommand{\tr}{\theta\sub{r}}
\newcommand{\sr}{\sigma\sub{r}}
\newcommand{\kg}{\kappa\sub{g}}
\newcommand{\tg}{\theta\sub{g}}
\newcommand{\sg}{\sigma\sub{g}}
\newcommand{\ky}{\kappa\sub{y}}
\newcommand{\ty}{\theta\sub{y}}
\newcommand{\sy}{\sigma\sub{y}}
\newcommand{\sry}{\sigma\sub{ry}}
\newcommand{\s}{\widehat s}
\newcommand{\hten}{$\hspace*{10mm}$}
\newcommand{\ti}{\tau\sub{i}}

%%JARROW YU
%\newcommand{\la}{\lambda\supp{1}}
%\newcommand{\lb}{\lambda\supp{2}}
%\newcommand{\lc}{\lambda\supp{3}}
\newcommand{\la}{\lambda\sub{1}}
\newcommand{\lb}{\lambda\sub{2}}
\newcommand{\lc}{\lambda\sub{3}}
\newcommand{\ga}{\gamma\sub{1}}
\newcommand{\gb}{\gamma\sub{2}}

%bansal-yaron
\newcommand{\sigc}{\sigma\sub{c}}
%%TABLES
\newcommand{\vh}{\vspace*{1cm}}

\title{{\bf How Large Can Jump-to-Default Risk Premia Be?\\
Modeling Contagion via the Updating of Beliefs.}\thanks{We
thank blah. This paper was previously circulated
under the name, ``Is Credit Event Risk Priced? Modeling Contagion
via the Updating of Beliefs.''} \vskip 20pt
%
\author{Pierre Collin-Dufresne\thanks{Carson Family Professor of Finance,
Columbia University, {\tt pc2415@columbia.edu}}
%
\hspace*{10mm}  Robert S. Goldstein\thanks{C. Arthur Williams Professor of
Insurance, University of Minnesota, golds144@umn.edu}
%
\hspace*{10mm}  Jean Helwege\thanks{Associate Professor at the Smeal
College of Business, Penn State University, University Park, PA 16802,
juh20@psu.edu}
%
%\hspace*{10mm} \thanks{} \\ \\
%{\bf \Large{Very Preliminary, Please do not quote.}}
}}

\date{First Version: January 28, 2002 \\
This Version:  December 20, 2008 \vspace{.6 in}\\
%{\bf Preliminary: Please do not quote}
}

\maketitle

\begin{abstract}

\begin{center}
{\bf \Large{How Large Can Jump-to-Default Risk Premia Be?\\
Modeling Contagion via the Updating of Beliefs.\\
{\tiny Previously entitled, ``Is Credit Event Risk Priced?
Modeling Contagion via the Updating of Beliefs.''}}}
\end{center}
%\vspace*{1.in}

Unable to explain sizable credit spreads through ``traditional
channels,'' reduced-form models of default often attribute a
significant portion of the credit spread to jump-to-default risk.
However, for reasons of tractability, these models preclude the most
likely justification for the jump-to-default event to be priced,
namely, a ``contagion-risk'' channel, where the aggregate corporate
bond index falls due to a credit event.  In this paper, we identify
a tractable framework for pricing corporate bonds in the face of
contagion-risk.  We argue that jump-to-default risk can explain only
a few basis points, as it must be dominated by contagion-risk.
Empirical support for these predictions is provided.

\end{abstract}

\newpage

%%%%%%%%%%%%%%%%%%%%%%%%%%%%%%%%%%%
%%%%%%%%%%%%%%%%%%%%%%%%%%%%%%%%%%%
%%%%%%%%%%%%%%%%%%%%%%%%%%%%%%%%%%%

\section{Introduction}

Most empirical studies of structural models of default have found
that only a small fraction of observed credit spreads for investment
grade debt can be explained in terms of compensation for credit
risk. (See Jones, Mason and Rosenfeld (1984), Huang and Huang
(2003), Eom, Helwege and Huang (2004)).
\nocite{jonmas84}\nocite{eomhel04}\nocite{huahua03} The problem is
especially severe for investment grade bonds with short maturities.
Indeed, if firm value dynamics are specified as a diffusion process,
then structural models predict a negligible default probability for
short maturity debt.

In contrast to structural models, reduced-form (or hazard rate)
models of default abstract from firm value dynamics and directly
model default as a jump event.  Reduced-form models posit a process
for the risk-neutral default intensity $\lambda^{Q}(t)$, and then value risky claims
by discounting at a default-adjusted rate under the risk-neutral
measure.  Under certain modeling assumptions, the price of a corporate
bond obtains the same analytic form as that found for a risk-free bond
(Lando (1998), Duffie and Singleton (1998)).  Indeed, it is this tractability
that explains the popularity of reduced-form models. However, the strength
of the reduced-form framework (i.e., its tractability and flexibility) is also its
weakness; because these models are primarily empirical, they make
few ``out-of-sample'' predictions about credit spreads. As such, they are as
good (or as bad) as their underlying assumptions regarding default intensity
and recovery rate dynamics.

Most sources of risk found in structural models have an analogue in
reduced form models.\footnote{Indeed, Duffie and Lando (2001)
demonstrate that when one adds uncertainty to the true firm value in
a structural model, the model effectively reduces to a reduced-form
model.} For example, just as common movements in firm values lead to
risk premia for corporate bonds in a structural
framework, common movements in intensities justify these same risk premia
in the reduced form framework, since such risks are not mitigated by
holding well-diversified portfolios. However, reduced-form models
have an additional channel for capturing compensation for risk not
found in (diffusion-based) structural models, namely, the
(unpredictable) jump-to-default event itself. When this risk is priced, the
risk-neutral intensity $\lambda^{Q}(t)$ is higher than the actual default intensity
$\lambda^{P}(t)$. Ignoring this additional risk factor, most empirical studies of
reduced form models face the same struggles as structural models do in justifying
observed credit spreads in terms of compensation for risk.
%\footnote{We note that even without the default event being priced,
%reduced-form models generate a non-zero credit spread at short
%maturities.  However, the size of the spread in that case is proportional
%to the probability of default over a short horizon, which
%historically has been extremely small.}
As such, these models tend to attribute a large proportion of the
credit spread to this jump-risk. For example, Driessen (2005)
estimates the jump-risk premium for 10-year BBB bonds in his
benchmark case to be 31bp, and the ratio of risk-neutral intensity to actual
default intensity $\left(\frac{\lambda^{Q}(t)}{\lambda^{P}(t)} \right)$ to be 2.3.
Similar results are reported by Berndt et al (2006), who instead
focus on credit default swaps.  We emphasize that these papers do not
estimate a jump-to-default premia directly.
Instead, researchers typically estimate the ``traditional'' channels directly,
and then {\em attribute the residual} to jump-to-default risk.  As such, obtaining an
accurate estimate for jump-to-default risk requires that researchers have a well-specified model.

However, we argue in this paper that these reduced form models
are not well-specified.  In particular, these models preclude
a ``contagion-risk'' channel, where the aggregate corporate
bond index falls due to a credit event.  Without such a channel, it
becomes difficult to justify why jump-to-default risk should be priced at all, since
it would otherwise appear to be diversifiable.  Moreover, we provide a simple equilibrium
framework that suggests the contagion risk premia must strongly dominate jump-to-default risk premia
for longer maturity bonds in economies with a large number of assets.  This is problematic,
since contagion-risk cannot explain short-maturity
spreads.  We are thus forced to conclude that short-maturity spreads are not due to jump-to-default risk,
but rather to non-credit factors, such as liquidity-risk.  Such conclusions have important
implications for optimal portfolio decisions, risk-management, and welfare concerns.

\ignore{
the most
likely justification for the jump-to-default event to be priced,
namely,   Below, we provide an equilibrium argument
that predicts the jump-to-default premium increases linearly with the
contagion premium.  Hence, if the contagion risk premium is zero, then so is the
jump-to-default risk premium.  We emphasize that this general result holds
regardless of whether the jumps considered are jumps-to-default, or simply jumps
in the risk-neutral default intensity (i.e., jumps in corporate bond prices).
}

%Moreover, we demonstrate that in an economy with many corporate bonds, the jump-to-default premium can
%explain only a few basis points, as it must be dominated by the contagion premium.
%Since jump-to-default risk is found to be small, our model implies that credit
%spreads at short maturities are mainly due to liquidity risk.\footnote{Taxes are another
%explanation for the existence of short maturity credit spreads (See, for example, Elton (2001)
%et al.  However, both Longstaff (2005) et al and Feldhutter and Lando (2007) find no evidence
%for a tax effect.  PIERRE, IS THIS TRUE? }

The reason that contagion risk has been mostly ignored in the reduced-form literature is not because
it has been deemed to be irrelevant, but rather because
accounting for it typically leads to an intractable framework.  Indeed, accounting for contagion
risk implies that the model falls outside of the ``no jump at default'' frameworks of Duffie,
Schroeder and Skiadis (199?) and Duffie and Singleton (199?), and also outside of the ``Cox-process''
framework described in Lando (1998).  As such, corporate bond prices no longer possess
simple analytic forms that risk-free bonds have.

Below, we provide a simple and intuitive framework that tractably captures
contagion risk regardless of the number of firms that share in the
contagion event. We propose a general reduced-form framework where an unexpected default of an
individual firm leads to a market-wide increase in credit spreads.  As such,
jump-to-default risk is not conditionally diversifiable, and hence commands a risk
premium.  While this framework is consistent with a counterparty-risk interpretation, it is most
naturally interpreted as an updating of beliefs about the economic environment due to
an unexpected default event.

There is a growing literature on the topic of event risk.
Conditions for which jump-to-default is not priced have
been investigated by Jarrow, Lando and Yu (2005).  Assuming a
doubly-stochastic Cox process framework (one where, conditional on
state variables driving the intensity processes, actual defaults are distributed as
independent, time-inhomogeneous Poisson processes) they show that default is
conditionally diversifiable (i.e., a large equally-weighted
portfolio of bonds is not affected by individual bond defaults). Thus,
in the limit when the number of firms becomes large,
jump to default risk-premium are not priced. This doubly
stochastic framework has been widely used in existing papers (Lando
(19??), Berndt et al. (), Driessen (), ...). Taking JLY's result at
face value, this is consistent with jump-to-default risk of a firm being
priced only if: i) this firm constitutes a significant fraction of the economy, ii) many
defaults occur jointly, or iii) marginal investors cannot diversify
their bond portfolio holdings. However, recent empirical findings question
the doubly-stochastic assumption.  For example, Das et al. (20??a,b)
report that the observed clustering of defaults in actual data are inconsistent
with this assumption.  Duffie et al (2008) use a fragility-based model similar to
ours to identify a hidden state variable consistent with a contagion-like response.
Jorion and Zhang (2007) blah blah.  Jarrow and Yu (2001) provide a model where
the default of one firm affects the intensity of another.  However, the model remains
tractable only for a ``small'' number $N$ of firms exposed to contagion-risk (e.g.,
JY investigate only $N=2$).  Such a model is not useful for studying the issue
at hand, namely, how large can the jump-to-default risk premium
be in an economy with a large number of firms issuing corporate bonds. Collin-Dufresne,
Goldstein and Hugonnier (2005) simplify the bond pricing formula of Duffie, Schroeder
and Skiadis (1996).  Note, however, that the formula itself does not identify a tractable framework for
pricing contagion risk.

Many other papers examine correlated default risk.  Duffie and
Singleton (1999) present various simulation techniques for
estimating correlation risk within reduced form models with
correlated intensities. Das, Freed, Geng and Kapadia (2006) provide
an empirical investigation of this type of intensity-based default
correlation.  Das, Duffie, Kapadia and Saita (2007) and
Duffie, Eckner, Horel and Saita (2007) show empirically that default
clustering exceeds the amount one would find in doubly stochastic models, instead arguing
for models that incorporate frailty, such as ours.\nocite {DDas, DEHS07}
Davis and Lo (2001) study a static model of
`infectious' default, which shares some of the notion of contagion
present in our framework, although it has no dynamic
updating of default probabilities. Zhou (2001)\nocite{zho01} and
Cathcart and El-Jahel (2002) consider structural models that allow
for jumps in firm value, but these are not very tractable for more
than two firms. Giesecke (2004) is closest to our paper in that
investors learn over time about the probability of default of one
firm from the defaults of other firms. However, in his model firms
can only default if they are currently at an all-time low price,
whereas in our framework credit events can occur at any time.
Our paper is also related to
work by Sch\"onbucher and Schubert (2001) who use a Copula
approach in a reduced-form framework.

Our approach parallels those in the learning and contagion
literature. In contrast to existing learning models (e.g., Detemple
(1986), Feldman (1989), David (1997) and Veronesi
(2000))\nocite{det86,fel89,ver00,dav97} that use results on
filtering theory for diffusions (see Liptser and Shiryaev
(1974)),\nocite{lipshi74bo} we derive results for continuous time
updating based on information revealed through point processes. This
allows learning when jumps occur rather than through diffusion-related defaults.

Our information-based mechanism for contagion is similar to that
proposed by King and Wadhwani (1990) and Kodres
and Pritsker (2002),\nocite{kodpri02} who investigate contagion
across international financial markets. There is also a large
empirical literature that studies contagion in equity markets (e.g.,
Lang and Stulz (1992)\nocite{lanstu92}) and in international finance
(e.g., Bae, Karolyi and Stulz (2003)).\nocite{baekar03} Theocharides (2007)\nocite{theo07}
investigates contagion in the corporate bond market and finds
empirical support for information-based transmission of crises.

As noted previously, the literature has estimated jump to default risk indirectly as
the residual of observed credit spreads minus what can be explained from traditional
channels.  The reason jump to default risk has not been investigated directly is because
investment grade firms almost never jump to default!! This fact alone should raise concern
about the magnitude of the jump-to-default premium.  Indeed, most papers associate
the historical one year default rate as a reasonable proxy for the instantaneous default rate,
but such a proxy dramatically overestimates the actual default probability over, say, a one month
period.  In our calibration below, we consider both jumps to default and the much more common
occurrence of a large jump in credit spreads (i.e., jump in risk-neutral intensity).  In both
cases, we conclude that the jump risk premium is necessarily quite small, as it is dominated
by the contagion risk premium.

Ultimately, it is an empirical question, whether event contagion is
present in the data. Therefore, we perform  an `event study' on a
large sample of corporate bond returns. We identify months that
include a surprise credit event (defined as a large jump in bond
prices, identified first by isolating large credit
spread jumps, and then confirmed by actual news event) of
an investment grade firm, and compare aggregated bond returns those
months to non-event months. We find that credit events lead to both market-wide
increases in credit spreads and to significant spill-over
returns in Treasury bonds, consistent with a  `flights to quality'
interpretation. The contagion effects are more pronounced for larger
firms than for smaller firms, and do not appear to be explained by a
host of control variables.

The rest of the paper is as follows.  In Section 2, we review the
relevant literature regarding jump to default risk, and explain why it is difficult
to capture contagion risk within a tractable framework.  In Section 3, we investigate
a simple production economy as in Ahn and Thompson (199?) to argue why contagion risk
may necessarily have to dominate jump to default risk in the presence of a large number
of firms.  In Section 4, we propose a model that captures contagion-risk in a tractable
framework, and demonstrate its implications for bond pricing and CDO pricing.  In Section 5,
we investigate empirically the impact that major credit events has had on the corporate bond
index. We conclude in Section 6.




%%%%%%%%%%%%%%%%%%%%%%%%%%%%%%%%%%%%%%%%%%%%%%%%%%%%%%%%%%%%%%%%%%%%%%%%%%%%%%%%%%%%%%
\section{Reduced Form Models: Background\label{dis}}
%%%%%%%%%%%%%%%%%%%%%%%%%%%%%%%%%%%%%%%%%%%%%%%%%%%%%%%%%%%%%%%%%%%%%%%%%%%%%%%%%%%%%%%

\subsection{Sources of risk}
In this section, we distinguish between jump-to-default risk and intensity risk
in reduced form models of default.  In the following section, we investigate situations
for which jump-to-default risk will be priced.

\subsubsection{Jump Risk}
Reduced form models of default\footnote{See, for example, Jarrow, Lando and Turnbull
(1995), Madan and Unal (1998), Duffie and Singleton (1999).} assume that default is
triggered by the jump of an unpredictable point process $\ind{\tau<t}$, where $\tilde{\tau}$ is
the random default time.  The
intensities under the historical probability measure $\lambda^{P}$ and risk neutral measure $\lambda^{Q}$
associated with this default event are defined via
\by\label{indp}
\mbox{E}_t^{P} \left[ d\ind{\tau\leq t} \right] &=& \lambda^{P}_t \, \ind{\tau >t} \, dt\\
%
\mbox{E}_t^{Q} \left[ d\ind{\tau\leq t} \right] &=& \lambda^{Q}_t \, \ind{\tau >t} \,
dt.
\ey
Intuitively, these equations imply that the probability of a jump to default during
the interval $(t, t + \Delta t)$, conditional upon no prior default, is $\lambda\sub{t} \, \Delta t$.

Regardless of whether a model is partial equilibrium (where the pricing kernel
is specified exogenously) or general equilibrium (where the pricing kernel is derived
endogenously from the agent's preferences and the technologies available),
if $d\ind{\tau\leq t}$ is contained in pricing kernel dynamics, then this source of risk is priced, and
$\lambda^{Q}\sub{t}$ will not equal $\lambda^{P}\sub{t}$.\footnote{This is the well-known result
of the change of measure, i.e., Girsanov's theorem for point processes. If the
Radon-Nykodim derivative has a common jump with the point process then its intensity may
be modified under the new measure. An example of this is
provided in equation~(\protect\ref{dxi}) and Lemma~\protect\ref{Qint} below.}
As a simple example, we specify the pricing kernel to be of the form
\by\label{sdf93}
\frac{d\Lambda}{\Lambda} = -r \, dt
+ \Gamma \left( d\ind{\tau\leq t} - \lambda^{P}_t \, \ind{\tau >t} \, dt \right).
\ey
Consider a risky bond with price $P(t)$ that has zero
recovery rate in the event of default. Its historical and risk-neutral dynamics are expressed as:
\by
\frac{dP}{P} &=& \mu \, dt - \left( d\ind{\tau\leq t} - \lambda^{P}_t \, \ind{\tau >t} \, dt \right) \\
%
&=& r \, dt - \left( d\ind{\tau\leq t} - \lambda^{Q}_t \, \ind{\tau >t} \, dt \right).
\ey
Combined, these equations imply:
\by\label{lamq76}
\left( \mu - r \right) = \left( \lambda^{Q} - \lambda^{P} \right) \ind{\tau >t}.
\ey

By definition of what a pricing kernel is, we find the relation
\by
0 &=& \frac{1}{dt} \mbox{E}^{P} \left[ \frac{dP}{P} + \frac{d\Lambda}{\Lambda}
+ \frac{dP}{P}\frac{d \Lambda}{\Lambda} \right] \nn \\
%
&=& \mu - r - \Gamma \lambda^{P} \ind{\tau >t}.
\ey
Combining this with equation~(\ref{lamq76}), we find
\by
\frac{\lambda^{Q}}{\lambda^{P}} &=& \left( 1 + \Gamma \right).
\ey
Interpreting, so long as $d\ind{\tau\leq t}$ shows up in the pricing kernel, that is, so
long as $\Gamma \neq 0$, then the ratio $\frac{\lambda^{Q}}{\lambda^{P}}$ will differ from unity,
and the compensation (i.e., expected excess return) for jump-to-default risk is
$\left( \mu - r \right) = \left( \lambda^{Q} - \lambda^{P} \right) \ind{\tau >t} = \Gamma \lambda^{P} \ind{\tau >t}$.
We emphasize that this channel has no analogue in (diffusion based) structural models of default.

\subsubsection{Intensity Risk}
In addition to jump-to-default risk,
the intensity $\lambda^{P}\sub{t}$ (and hence, in general,  also $\lambda^{Q}\sub{t}$) itself is specified
to have stochastic dynamics, capturing the fact that the likelihood of default changes over
time. In particular, its dynamics can change randomly due to both Brownian motions $z$
and jumps $q$:
\by\label{lam_p_12}
d\lambda^{P}_t &=& \mu^{P}\sub{\lambda} \, dt + \sigma\sub{\lambda}
dz^{P}_t + \widetilde{\Gamma\sub{\lambda}} \, dq_t. \nonumber\\ \label{lamq}
%
&=& \mu^{Q}\sub{\lambda} \, dt + \sigma\sub{\lambda} dz^{Q}_{t} +
\widetilde{\Gamma\sub{\lambda}} \, dq_{t}.\label{lam_q_12}
\ey
If changes in $\lambda^{P}$ are correlated with
changes in the pricing kernel, then at least one of the sources of risk $dz$ and $dq$ is priced, and
the {\sl dynamics} for $\lambda^{P}$ will differ under the historical and risk-neutral
measures.

As equations~(\ref{indp})-(\ref{lamq}) suggest, within a reduced-form framework, risk
premia can show up in two different manners.  First, risk premia can be due to sources
of risk that drive the dynamics of the intensity ($dz^{P}$, $dq$).
These risk sources have an analogue in structural models of default, where
($dz^{P}$, $dq$) would drive the dynamics of distance-to-default.
Duffee (2002) and Driessen (2005) provide convincing evidence that such
risk premia exist in reduced form models, whereas
Elton et al.\ (2001) and Collin-Dufresne, Goldstein and Martin (2001)
provide similar evidence that such risk premia exist in structural models.
Second, the jump-to-default random variable $d\ind{\tau\leq t}$ can command a risk premium itself,
in which case $\lambda^{P} \neq \lambda^{Q}$.  This source of risk has no analogue in
(diffusion-based) structural models of default, and hence can {\em potentially} help explain
the empirical failures of structural models.
We emphasize that these risk premia are empirically distinguishable, both in time-series and
in cross-section.  In time series,
for example, $d\ind{\tau\leq t}$ is priced only if there is a market-wide response at
the default event. In cross-section, abstracting from taxes and liquidity, only jump-to-default
risk can generate credit spreads that are higher than expected loss rates at the very
short end of the yield curve.
%Driessen (2002) indirectly estimates this jump risk premium
%by estimating the risk premia due to many other sources (taxes, liquidity, etc.), and
%interprets the residual as jump-to-default risk.

%Huang and Huang (2002) document that highly-rated corporate bonds trade at very large
%spreads relative to Treasuries even though their historical default probabilities are
%very low.  This observation is consistent with the belief that jump-risk carries a substantial
%risk-premium. Indeed, Driessen (2002) estimates the risk-neutral default intensity to
%be two- to six-times its historical counterpart to reconcile prices of corporate bonds with
%historical default and recovery rates.

\subsection{Conditions for Jump-to-Default Risk to be Priced}

In equation~(\ref{sdf93}), we have simply assumed that jump-to-default risk is priced.
One of the major issues in this paper is to investigate the conditions for this risk to be priced
for the typical corporate bond.  Recently, there has been considerable research on this topic.
For example, by extending the arguments implicit in the APT framework,
Jarrow, Lando and Yu (2001) discuss the conditions for which no systematic
jump-to-default-risk exists.  Essentially, their results show that if the following
two conditions are satisfied:
\begin{tabbing}
%\hten
\hspace*{5mm} \=(i) \=Conditional on the state variables driving intensities, default events
%\\ \> \>
are independent. \\
\>(ii) A large number of bonds are available for trading,
%CITE THE STANFORD GRADUATE STUDENT HERE
\end{tabbing}
then jump-risk is conditionally diversifiable, and therefore should not command
a risk-premium.

There are at least two different scenarios where we can expect jump-risk to be priced.\nocite{jarlan00}
First, there can be {\em systemic risk} in the sense that
firms default at the same time (i.e., $d\ind{\tau\sub{i}\leq t} \, d\ind{\tau_{j}\leq t} \neq
0\,\,\,\forall\,\, i,j\in\,\, [1,N]$).
Intuitively, if the number of firms $N$ is large enough that a non-negligible part of the
economy defaults at the same date, then such a
risk would command a risk-premium. However, there is little empirical support for such a
notion. (Of course, there is always the concern of a `Peso-problem').

Second, there can be {\em contagion-risk} in the sense that the default of one firm
can trigger an increase in the risk (i.e., an increase in the intensity) of default of
other firms. Mathematically, we can write this as
$d\ind{\tau_i\leq t} \, d\lambda_j(t)\neq 0\,\,\,\forall\,\,i,j\,\,\in[1,N]$.  In this paper,
we focus on contagion risk as the justification for jump-to-default risk to be priced.

Unfortunately, accounting for contagion risk destroys the tractability of the previously
proposed models in the literature. Indeed, the empirical literature has focused on those models where
the zero-recovery, zero-coupon risky bond price
\by\label{rnm65}
P^{T}(0) = \mbox{E}^{Q}\sub{0} \left[ e^{-\int_{0}^{T} ds \, r(s)} \, \ind{\tau >  T} \right]
\ey
can be re-expressed as
\by\label{rf65}
P^{T}(0) = \mbox{E}^{Q}\sub{0} \left[ e^{-\int_{0}^{T} dt \, \left( r(t) + \lambda^{Q}(t) \right)}  \right].
\ey
However, as noted by Duffie, Schroeder and Skiadas (199?), Duffie and Singleton (1998) and Lando (1998),
when there is contagion-risk, then equation~(\ref{rnm65}) is typically not equal to equation~(\ref{rf65}).
To understand why, first consider a model with a trivial sense of contagion, namely, where
a firm's default causes a jump in its own intensity. Admittedly, this model is not
intuitively satisfying, but it has the advantage of clearly demonstrating the mathematically relevant issues.
In particular, consider a simple economy where the
risk free rate is constant, and a firm's risk-neutral intensity is a constant except at
the default event:
\by
d \lambda^{Q}(t) = \alpha \, d\ind{\tau\leq t}
\ey
Consider a discount bond with zero recovery.  Its price can be represented as
\by
P^{T}(0) = \mbox{E}^{Q}\sub{0} \left[ e^{-r T} \, \ind{\tau >  T} \right].
\ey
The solution to this expectation can be obtained by noting that in each period $(t, t + dt)$, the probability
that the firm does not jump to default is $e^{-\lambda^{Q}\sub{0} \, dt}$.  Since survival implies
not defaulting at each interval $dt$, piecing all periods together, we
find
\by\label{bond38}
P^{T}(t) = e^{- (r + \lambda^{Q}\sub{0}) T}
\ey
Note that this solution is not equal to that obtained when we solve the expectation in equation~(\ref{rf65}).
Indeed, using the expectations
\by
\mbox{E}^{Q}\sub{0} \left[\ind{\tau >  T} \right] &=& e^{-\lambda^{Q}\sub{0} T}, \nn \\
%
\mbox{E}^{Q}\sub{0} \left[\ind{\tau  \in (t, t + dt)} \right] &=& e^{-\lambda^{Q}\sub{0} t} \, \lambda^{Q}\sub{0} \, dt,
\ey
we find
\by
\mbox{E}^{Q}\sub{0} \left[ e^{-\int_{0}^{T} dt \, \lambda^{Q}(t) }  \right] &=&
\left( e^{-\lambda^{Q}(0) T} \right) \left(e^{-\lambda^{Q}\sub{0} T} \right)
+ \int_{0}^{T} dt e^{-\lambda^{Q}\sub{0} t} \, \lambda^{Q}\sub{0}
e^{-\lambda^{Q}\sub{0} u} e^{- (\lambda^{Q}\sub{0}+ \alpha) (T-u)} \nn \\
%
&=& e^{-2\lambda^{Q}\sub{0} T}
+ \left( \frac{\lambda^{Q}\sub{0}}{\lambda^{Q}\sub{0} - \alpha }\right) e^{-(\lambda^{Q}\sub{0} + \alpha) T}
\left[ 1 - e^{-(\lambda^{Q}\sub{0} - \alpha) T} \right]. \label{lam_int}
\ey
Note that only in the case $\alpha = 0$, which refers to the case where the intensity is not
affected by the default event, does equation~(\ref{lam_int}) reduce to equation~(\ref{bond38}).

A more interesting model which captures the notion of contagion is the
(N = 2) counterparty-risk model of Jarrow and Yu (JY 2001):
\begin{eqnarray}\label{JY2a}
\la(t) &=& a\sub{11} + a\sub{12} \, \ind{\tau^{2}\leq t} \\ \label{JY2b}
%
\lb(t) &=& a\sub{21}\, \ind{\tau^{1}\leq t} + a\sub{22} .
\end{eqnarray}
Intuitively, this model captures the notion that a default of one firm affects the intensity/probability
of future default of another firm. While analytic solutions exist for bond prices have been
identified for $N=2$, such solutions quickly become intractable as $N$ becomes large,
even for the simplest case where the coefficients $\{ a\sub{ij}\}$ are deterministic constants.
As in our example above, the prices of these bonds cannot be expressed as
\by
P^{T}(t) \neq \mbox{E}^{Q} \left[ e^{-\int_{t}^{T} \, ds \,(r(s) + \lambda^{Q}(s)) } \right].
\ey

Below, we propose a model that can capture contagion across an arbitrarily large number of firms.
While our model can be interpreted as capturing counter-party risk as in JY,
it will be most natural to interpret it as capturing contagion through Bayesian
updating of beliefs.  Most importantly, it will provide a tractable framework even when the
number of firms sharing in the contagion is large.  Before we introduce the model, however, we
provide a simple framework that suggests contagion-risk premia will dominate jump-to-default premia.


\ignore{
The most natural economic motivation for the JY model of cross-dependent contagion
is that firms have economic ties that render each firm vulnerable to the default of
the other. While such mechanism is perhaps appropriate for firms with economic ties such as
supplier-producer relations, it would not appear able to explain common jumps in spreads across
bonds of various industries, such as what happened after the RJR and Enron credit
events.  To provide a mechanism for that type of an example, here we model contagion through Bayesian
updating of beliefs.  We first motivate the dynamics of the model using a simple example
driven by a single state variable. Then, we generalize the framework to multiple
state variables.
}


\section{Comparing Jump Risk Premia and Contagion Premia}

In this section, we investigate within a general equilibrium production economy the
relative sizes of jump risk premia and contagion premia when there are a large number of
``firms'' $N$.  Recall that JLY used APT-like arguments to show that if the jump-risk is conditionally
diversifiable, and $N$ becomes large, then the jump risk premia goes to zero. Here, we
generalize their results by allowing for contagion risk.  As such, jump-risk will not
conditionally diversifiable here.

We consider a production economy with linear technologies as in Cox, Ingersoll and Ross (1985) and
Ahn and Thompson (1988).  For reasons of tractability, we consider $N$ identical productive technologies
with return dynamics:
\by
\frac{dS\sub{i}}{S\sub{i}}= \mu \, dt + \sigma\sub{0} \, dz  + \sigma\sub{i} \, dz\sub{i}
- \Gamma\sub{D} \left( dq\sub{i} - \lambda^{P} \, dt \right)
- \Gamma\sub{C} \sum_{j\neq i}^{N} \left( dq\sub{j} - \lambda^{P} \, dt \right)
\ey
where $dz$ is a Brownian motion common to all firms, the $\{dz\sub{i}\}$ are idiosyncratic
Brownian motions orthogonal to each other and to $dz$, and
the $\{ dq \}$ are Poisson random variables with intensity $\lambda^{P}$.
The term $\Gamma\sub{D} dq\sub{i}$ is meant to capture a ``credit event'' associated with
firm-$i$, whereas the term $\Gamma\sub{C} dq\sub{j \neq i}$ is meant
to capture firm-$i$'s contagious response to a ``credit event'' associated with
investment-$j$.  As such, we shall calibrate $\Gamma\sub{D} > \Gamma\sub{C}$.  That is, we expect
the credit event associated with firm-$i$ to affect returns on firm-$i$ significantly more than
the returns on the overall index.


Assume that the representative agent in the economy maximizes her expected utility:
\bq
J(W,s) = \max\E\left[\int_s^\infty e^{-\delta t} \, \frac{C(t)^{1-\gamma}}{1-\gamma} dt\right]
\eq
subject to her budget constraint
\by
dW\sub{t} &=&
\left\{ rW\sub{t} - C\sub{t} + W\sub{t} \left( \sum_{i=1}^{N} \pi\sub{i} \right)
\left[ \big (\mu -r) + \lambda^{P} \left( \Gamma\sub{D} + (N-1) \Gamma\sub{C} \right) \right] \right\} \, dt
+ W\sub{t} \sigma\sub{0} \left( \sum_{i=1}^{N} \pi\sub{i} \right) \, dz \nn \\
%
&&
+ W\sub{t} \sigma\sub{1} \left( \sum_{i=1}^{N} \pi\sub{i} \, dz\sub{i} \right)
- W\sub{t} \left( \Gamma\sub{D} - \Gamma\sub{C} \right)  \left( \sum_{i=1}^{N} \pi\sub{i} \, dq\sub{i} \right)
- W\sub{t}  \Gamma\sub{C}  \left( \sum_{i=1}^{N} \pi\sub{i} \right) \sum_{j=1}^{N}\, dq\sub{j}.
\ey
Here, we have defined $\pi\sub{i}$ to be the proportion of wealth placed into investment
technology $i$.

Applying Ito's lemma, the value function $J(W,t)$ satisfies the HJB equation:
\by
0 &=& \max_{C,\pi} \left\{  e^{-\rho t} \frac{C^{1-\gamma}}{1-\gamma}  + J\sub{t} +
J\sub{W} \left[ rW\sub{t} - C\sub{t} + W\sub{t} \left( \sum_{i=1}^{N} \pi\sub{i} \right)
\left[ \big (\mu -r) + \lambda^{P} \left( \Gamma\sub{D} + (N-1) \Gamma\sub{C} \right) \right] \right]
\right. \nn \\
%
&& \hspace*{10mm}
+ \frac{1}{2} W^{2} J\sub{WW} \left[ \sigma\sub{0}^{2}\left( \sum_{i=1}^{N} \pi\sub{i} \right)^{2}
+ \sigma\sub{1}^{2} \left( \sum_{i=1}^{N} \pi\sub{i}^{2} \right) \right] \nn \\
%
&& \left. \hspace*{10mm} + \lambda^{P} \sum_{i=1}^{N}
\left[ J\left( W - W(\Gamma\sub{D} -\Gamma\sub{C} ) \pi\sub{i} - W \Gamma\sub{C}
\left( \sum_{i=1}^{N} \pi\sub{i} \right) \right) - J(W) \right] \right\}.
\ey
\ignore{
%
&&
x'\Omega x+J_W \left(rW_t-C_t+ W_t x'(\mu-r\One)\right)\right.\nonumber\\
 &&\left.+ \sum_i \lambda^{P}_i \left[J\left(W(1+\gb x'\One+ (\ga-\gb)x_i)\right)-J(W)\right]\right\}\ey
 }
The first order condition are:
\by
\frac{\partial}{\partial C}: \qquad 0 &=& e^{-\rho t} C\sub{t}^{-\gamma} - J\sub{W} \\
%
\frac{\partial}{\partial \pi\sub{k}}: \qquad
0 &=&
W\sub{t} J\sub{W} \left[ (\mu -r) + \lambda^{P} \left( \Gamma\sub{D} + (N-1) \Gamma\sub{C} \right) \right]
+ W^{2} J\sub{WW} \left[ \sigma\sub{0}^{2} \left( \sum_{i=1}^{N} \pi\sub{i} \right)
+ \sigma\sub{1}^{2} \pi\sub{k} \right]  \\
%
&& - \lambda^{P} W \sum_{i=1}^{N} \left\{
J'\left( W - W(\Gamma\sub{D} -\Gamma\sub{C} ) \pi\sub{i} - W \Gamma\sub{C}
\left( \sum_{j=1}^{N} \pi\sub{j} \right) \right)
\left[ \left( \Gamma\sub{D} - \Gamma\sub{C} \right) {\bf 1}\sub{(i=k)} - \Gamma\sub{C} \right]. \nn
\right\}
\ey
Due to the symmetric nature of all returns, and due to the fact that in equilibrium the risk free
rate $r$ will adjust so that the bond is held in zero net supply, it follows that the agent will place
a constant fraction of her wealth in each investment technology:  $\pi\sub{i} = \frac{1}{N} \; \forall i$.

It is well known that the solution to the indirect utility function takes the form
\by
J(W,t) = e^{-\rho t} A^{-\gamma} \frac{W^{1-\gamma}}{1-\gamma}.
\ey
Plugging this into the consumption first order condition, we see that the price-consumption
ratio is a constant
\by
C &=& A W.
\ey
More relevant for the issue at hand, we find that the equilibrium excess return for each
investment technology follows
\by\label{excess86}
\left( \mu - r \right) &=& \gamma \left( \sigma\sub{0}^{2} + \frac{\sigma\sub{1}^{2}}{N} \right)
+ \lambda^{P} \left[ \Gamma\sub{D} + (N-1) \Gamma\sub{C} \right]
\left\{ \left[ 1- \frac{1}{N} \left( \Gamma\sub{D} + (N-1) \Gamma\sub{C} \right) \right]^{-\gamma} - 1 \right\}.
\ey
Further, from equation~(\ref{lamq76}), we see that the ratio of risk-neutral and actual intensity
$\left( \frac{\lambda^{Q}}{\lambda^{P}} \right)$ can be expressed as
\by
\left( \frac{\lambda^{Q}}{\lambda^{P}} \right) &=&
1 + \left[ \Gamma\sub{D} + (N-1) \Gamma\sub{C} \right]
\left\{ \left[ 1- \frac{1}{N} \left( \Gamma\sub{D} + (N-1) \Gamma\sub{C} \right) \right]^{-\gamma} - 1 \right\}.
\ey
PIERRE, IS THIS CORRECT?

Equation~(\ref{excess86}) has a straightforward interpretation:  The first term on the right
hand side is the contribution to
excess return due to the diffusion sources of risk $(dz, \, \{dz\sub{i}\} )$:
\by
\left( \mu - r \right)\sub{diffusion} &=&
\gamma \left( \sigma\sub{0}^{2} + \frac{\sigma\sub{1}^{2}}{N} \right).
\ey
As in the standard diffusive model, the diffusion premium is a
combination of relative risk aversion $\gamma$ and market variance.
Consistent with standard APT arguments, we see that the idiosyncratic diffusive component
$ \left( \gamma \frac{\sigma\sub{1}^{2}}{N} \right)$ becomes negligible as the number of
firms $N$ increases.

The second term on the right hand side is the contribution of
excess return due to the jump sources of risk $( \{dq\sub{i}\} )$. It is useful to
decompose this second term into a jump-risk component and a contagion-risk component:
\by\label{jump_comp}
\left( \mu - r \right)\sub{jump} &=& \lambda^{P} \Gamma\sub{D}
\left\{ \left[ 1- \frac{1}{N} \left( \Gamma\sub{D} + (N-1) \Gamma\sub{C} \right) \right]^{-\gamma}
- 1 \right\}\\
%
\left( \mu - r \right)\sub{contagion} &=& \lambda^{P} (N-1) \Gamma\sub{C}
\left\{ \left[ 1- \frac{1}{N} \left( \Gamma\sub{D} + (N-1) \Gamma\sub{C} \right) \right]^{-\gamma}
- 1 \right\}.\label{cont_comp}
%
\ey
It is worth noting that when $\Gamma\sub{C}$ is finite, then a jump-event leads to
a market-wide response in returns, and both components contribute to investment returns.
However, if the contagion risk is zero, $\Gamma\sub{C} = 0$, then not only is the
contagion-risk zero (by definition), but the jump-component of excess return also
vanishes as $N$ becomes large:
\by
\left. \left(\mu - r\right)\right|\sub{jump, \Gamma_{C} = 0} &=& \lambda^{P} \Gamma\sub{D}
\left\{ \left[ 1- \frac{\Gamma\sub{D}}{N} \right]^{-\gamma} - 1 \right\} \nn \\
%
&\stackrel{N\Rightarrow \infty}{\approx}& \left(\frac{1}{N} \right) \lambda^{P} \gamma \Gamma^{2}\sub{D}.
\ey

Now, we calibrate this model using equation~(\ref{excess86}).  We emphasize that, up
to this point, we did not limit these jumps to be either jumps to default, or simply just
jumps in credit spreads.  Historically, recovery rates on corporate bonds has been approximately
40\% for senior secured debt.  Hence, when investigating jump-to-default, we set
$\Gamma\sub{D} = 0.6$.  Separately, in our empirical section, we investigate credit events where
a firm's bond price fell on average by 15\%, so when studying that case, we shall set $\Gamma \approx .15$.
Throughout, we shall investigate $N = 1000$, since there are currently
approximately 1000 firms with investment-grade status.

Historically, the non-callable BBB-Treasury spread on 4-year debt has been estimated
to be about 150bp (See, for example, Huang and Huang (2003), Chen et al (2007)).  Approximately
25bp of that spread covers historical losses, so the risk premium has been approximately 125bp.
Here, we look for an upper bound for the jump-to-default risk premium, so we set the diffusion
premium to zero.  Thus, the entire 125bp risk premium is being attributed to contagion risk and jump-to-default
risk only.  In Table 1, we decompose the risk premium for these two components
for different values of the risk aversion coefficient $\gamma$ for an economy with $N = 1000$
investment technologies. In particular, using equation~(\ref{hi}) and setting the
volatility components to zero, we identify the implicit value for the contagion risk parameter
$\Gamma\sub{C}$, and in turn, the contagion premium and jump-to-default premium.  To calibrate
this model, we use the historical one year default rate
for BBB rated bonds from 1920-2004 of 0.3\% obtained from Moody's.  Note that this number greatly
overestimates the one-month intensity.
Indeed, we are aware of only 5 firms that have defaulted with investment-grade status.
Furthermore, the majority of these did not possess investment-grade spreads -- that is, the market
realized that these bonds were not truly investment-grade, and that the rating
agencies were slow to downgrade.  Still, to
provide an upward bound, we will choose $\lambda^{P} = 0.003$.

We see that, even for a relatively
high risk aversion coefficient $\gamma = 20$, we can attribute only 5.6bp of risk premium to jump-to-default
risk.  Accounting for diffusion risk, liquidity risk, taxes, etc., will reduce this number further.
Indeed, if instead of (BBB-Treasury) spread, we begin with credit default swap data, which
reduces the spread to approximately 100bp, and then subtract 25bp for expected losses and 20bp for diffusion
risk, then we have 55bp of ``unexplained risk premium''.  Under this more realistic calibration,
we find that the premium due to contagion risk varies from 0.6bp to 1.2bp as we go from $\gamma = 5$ to
$\gamma = 20$.  The bottom line is that there appears to be no calibration possible where jump to
default risk is large when the number of firms $N$ is large.  Instead, contagion risk must make up
almost the entire unexplained risk premia.

However, this leads to a problem: whereas jump-to-default risk generates
sizeable credit spreads a short maturities, contagion-risk premia do not.
Since investment-grade credit spreads are rather large at short maturities, this forces
us to conclude that a sizeable fraction of spreads, (even CDS spreads !!) is due to non-credit
factors such as liquidity.\footnote{Of course, accounting for liquidity risk will reduce the amount of
remaining spread to be explained, in turn reducing that part due to jump-to-default risk even
further.}

The interpretation of these results is straightforward:  from equations~(\ref{jump_comp})
-(\ref{cont_comp}), the relative size of the jump component to the contagion component
is expressed by the ratio $\frac{\Gamma\sub{D}}{(N-1) \Gamma\sub{C}}$.  Although we find
$\Gamma\sub{D} \gg \Gamma\sub{C}$, that is, the firm suffering the credit event performs
much worse than the index which shares in the contagion, $\Gamma\sub{D} \ll (N-1) \Gamma\sub{C}$.
As such, the contagion risk premia dwarfs the jump-to-default premium.  Interestingly, the range
of ratios $\frac{\Gamma\sub{D}}{\Gamma\sub{C}} \in (22, 47)$ found in this example for different levels
of $\gamma$ matches well our empirical findings discussed below.
Interpreting, while it is about $\frac{\Gamma\sub{D}}{\Gamma\sub{C}} \approx 30$-times worse
to be impacted by a ``credit event'' than to just be part of the contagion-risk, it is
$N = 1000$-times more likely to be part of the contagion risk.

\begin{table}\footnotesize \renewcommand{\baselinestretch}{0.8}
\begin{center}
\begin{tabular}{ccccc}\hline
$\gamma$    &   $\Gamma^{implied}\sub{C}$   & $(\mu - r)\sub{C}$  &
$(\mu - r)\sub{J}$   &   $\left( \frac{\lambda^{Q}}{\lambda^{P}} \right)$    \\ \hline
$\;$5   &   0.027   &  122.3   &   2.7 &   1.27    \\
10  &   0.019   &  121.1   &   3.9 &   1.39    \\
15  &   0.015   &  120.2   &   4.8 &   1.48    \\
20  &   0.013   &  119.4   &   5.6 &   1.56    \\ \hline
\end{tabular}
\end{center}
\caption{\footnotesize  We decompose 125bp of excess return into that due to jump-to-default
risk and that due to contagion risk for select values of the constant relative risk aversion
coefficient $\gamma$.  Parameter values are $\Gamma\sub{D} = 0.6$,
$\lambda^{P} = 0.003$ and $N = 1000$. }
\end{table}

\section{A Tractable Model of Contagion-Risk}
\renewcommand{\lb}{\overline{\lambda}}

In the previous section, we have provided a very general framework that suggests
contagion risk premia should dwarf jump-to-default risk premia.  Here, we focus
on the pricing of risky bonds.  We provide a very tractable framework for pricing
risky debt in the presence of contagion-risk, even if many firms share in the contagion
risk.

Consider an economy where the true state of nature $\tilde{S}$ is
unknown and can be in any one of $j \in(1,J)$ states.  At date-$t$,
investors do not know what state the economy is in, but form a prior
$p\sub{j}(t) \equiv \mbox{Prob}(\tilde{S} = j | \F\sub{t})$, where
$\F\sub{t}$ is the investors' information set at date-$t$. In this
economy there are $N$ defaultable firms indexed by $i \in(1,N)$ with
random default times $\tau\sub{i}$ driven by point processes
characterized by default intensities.  In particular, conditional
upon being in state-$j$, the probability of default over the next
interval $dt$ is expressed via \bq\label{lamij} \mbox{E}\left[\left.
d\ind{\ti<t} \right| \tilde{S} = j, \, \F\sub{t} \right] =
\lambda\sub{ij}(t^{-}) \, \ind{\ti>t} \, dt. \eq That is, we can
interpret $\lambda\sub{ij}(t^{-})$ as the date-$t$ default intensity
for firm-$i$ conditional upon being in state-$j$.  Below, we will
assume that, conditioning both on the state-$j$ and the paths
$\left. \lambda\sub{ij}(t^{-})\right|_{t=0}^{T}$ for some distant
future date-$T$, the default events across firms are independent. In
technical terms, we are assuming a doubly-stochastic, or Cox-process
{\em conditional} upon being in a particular state=$j$.
\footnote{See, for example, Lando (1998)} We emphasize, however,
because agents do not know the correct state-$j$, our model falls
outside of the Cox-process framework, as will be made clear below
in, for example, equation~(\ref{nocox}).

Since investors do not know the actual state of nature, their
estimate of the actual default intensity $\lb\sub{i}(t^{-})$ is
defined implicitly through \by \lb\sub{i}(t^{-}) \, \ind{\ti>t} \,
dt &\equiv&
\mbox{E} \left[ \left. d\ind{\ti<t} \right| \F\sub{t} \right]  \nn \\
%
&=& \sum_{j=1}^{J} p\sub{j}(t) \, \mbox{E} \left[ \left.
d\ind{\ti<t} \right| \tilde{S} = j, \, \F\sub{t} \right]
\nn \\
%
\label{lambar2} &=& \sum_{j=1}^{J} p\sub{j}(t)
\lambda\sub{ij}(t^{-}) \, \ind{\ti>t} \, dt. \ey That is, from the
information set of the investor, the default intensity
\bq\label{lambar} \lb\sub{i}(t^{-}) = \sum_{j=1}^{J} p\sub{j}(t)
\lambda\sub{ij}(t^{-}) \eq is simply a weighted average of the
conditional default intensities.

We assume that investors continuously update their estimates of the
$\{p\sub{j}(t)\}$ conditional upon whether or not they observe a
default event during the interval $dt$. Here we provide a heuristic
derivation for the updating process.  In Appendix A we provide a
rigorous proof.

Since defaults are triggered by point processes, investors observe,
{\em at most}, one event per unit time.  Define $d{\bf 1}\sub{t}
\equiv \left. d\ind{\tau\sub{i} < t}\right|_{i=1}^{N}$ as the vector
of jump events. Consider first the case where no default is observed
in a period $dt$. Using the definition of conditional probability,
and keeping terms only to ${\cal O}(dt)$, we obtain: \by \Pr \left[
\left. \tilde{S} = j, \, \dq\sub{t}=0 \right| \F\sub{t} \right] &=&
\Pr \left[ \left.  \dq\sub{t}=0 \right| \tilde{S} = j, \, \F\sub{t}
\right] \times \Pr \left[ \left.  \tilde{S} = j \right| \F\sub{t}
\right]
\nonumber \\
%
&=& \prod_{i=1}^{N} \left( 1 - \lambda\sub{ij}(t^{-}) \ind{\ti>t}dt
\right) \times
p\sub{j}(t)  \nonumber \\
%
&\stackrel{{\cal O}(dt)}{=}& p\sub{j}(t) \, \left( 1-\sum_{i=1}^{N}
\lambda\sub{ij}(t^{-}) \ind{\ti>t}dt \right). \ey

Again using conditional expectations, we find that the process for
$p\sub{j}(t)$, {\em conditional} upon no firm defaulting during the
interval $(t, t + dt)$, evolves via \by \left.
p\sub{j}(t+dt)\right|_{\dq\sub{t} = 0} &\equiv& \Pr \left[ \left.
\tilde{S} = j \right|  \dq\sub{t}=0, \, \F\sub{t} \right]
\nonumber \\
%
&=& \frac{\Pr \left[ \left. \tilde{S} = j, \, \dq\sub{t}=0 \right|
\F\sub{t} \right]}
{\Pr \left[ \left. \dq\sub{t}=0 \right| \F\sub{t} \right]} \nonumber \\
%
&=& \frac{\Pr \left[ \left. \tilde{S} = j, \, \dq\sub{t}=0 \right|
\F\sub{t} \right]} {\sum_{j'=1}^{J} \Pr \left[ \left. \tilde{S} =
j', \, \dq\sub{t}=0 \right|
\F\sub{t} \right]} \nonumber \\
%
&=& \frac{p\sub{j}(t) \, \left( 1-\sum_{i=1}^{N}
\lambda\sub{ij}(t^{-}) \ind{\ti>t}\,dt \right)} {\sum_{j'=1}^{J}
p\sub{j'}(t) \, \left( 1-\sum_{i=1}^{N} \lambda\sub{ij'}(t^{-})
\ind{\ti>t}\,dt \right)} \nonumber \\
%
&=& \frac{p\sub{j}(t) \, \left( 1-\sum_{i=1}^{N}
\lambda\sub{ij}(t^{-}) \ind{\ti>t}\,dt \right)}
{1-\sum_{i=1}^{N} \overline{\lambda}\sub{i}(t^{-}) \ind{\ti>t}\,dt} \nonumber \\
%
&\stackrel{{\cal O}(dt)}{=}& p\sub{j}(t) \left[ 1 - \sum_{i=1}^{N}
\left( \lambda\sub{ij}(t^{-}) - \overline{\lambda}\sub{i}(t^{-})
\right) \ind{\ti>t} \, dt \right], \ey where we have used
equation~(\ref{lambar}) and the fact that $\sum_{j'=1}^{J}
p\sub{j'}(t) = 1$ in the second-to-last line.  Hence, if there are
no jumps during the interval $dt$, then the Bayesian updating
follows the process: \bq\label{nodef} \left.
dp\sub{j}(t)\right|_{\dq\sub{t} = 0} = - p\sub{j}(t) \sum_{i=1}^{N}
\left( \lambda\sub{ij}(t^{-}) - \overline{\lambda}\sub{i}(t^{-})
\right) \ind{\ti>t}\,dt. \eq

In contrast, if {\em one} firm (e.g., firm i) defaults during the
interval $dt$, then, for all states-$j$, we obtain the updating: \by
\Pr \left[ \left. \tilde{S} = j, \, d\ind{\tau\sub{i} < t}=1 \right|
\F\sub{t} \right] &\equiv& \Pr \left[ \left.  d\ind{\tau\sub{i} <
t}=1 \right| \tilde{S} = j, \, \F\sub{t} \right] \times \Pr \left[
\left. \tilde{S} = j \right| \F\sub{t} \right]
\nonumber \\
%
\label{update44} &=& p\sub{j}(t) \, \lambda\sub{ij}(t^{-}) \,dt. \ey

Hence, it follows that, {\em conditional} on firm-$i$ defaulting
during the interval $(t, t + dt)$, the process for $p\sub{j}(t)$
evolves via \by \left. p\sub{j}(t+dt)\right|_{d\ind{\tau\sub{i} <
t}=1} &\equiv& \Pr \left[ \left.  \tilde{S} = j \right|
d\ind{\tau\sub{i} < t}=1 , \, \F\sub{t}\right]
\nonumber \\
%
&=& \frac{\Pr \left[ \left. \tilde{S} = j, \, d\ind{\tau\sub{i} <
t}=1 \right| \F\sub{t} \right]} {\Pr \left[ \left. d\ind{\tau\sub{i}
< t}=1 \right| \F\sub{t} \right]}
\nonumber \\
%
&=& \frac{\Pr \left[ \left. \tilde{S} = j, \, d\ind{\tau\sub{i} <
t}=1 \right| \F\sub{t} \right]} {\sum_{j'=1}^{J} \Pr \left[ \left.
\tilde{S} = j', \, d\ind{\tau\sub{i} < t}=1
\right| \F\sub{t} \right]} \nonumber \\
%
&=& \frac{p\sub{j}(t) \, \lambda\sub{ij}(t^{-}) }
{\overline{\lambda}\sub{i}(t^{-})}, \ey where we have used
equations~(\ref{lambar}) and (\ref{update44}) in the last line.
Therefore, we can write \by \label{def2} \left.
dp\sub{j}(t)\right|_{d\ind{\tau\sub{i} < t}=1} &=& p\sub{j}(t) \,
\left(
\frac{\lambda\sub{ij}(t^{-})}{\overline{\lambda}\sub{i}(t^{-})} - 1
\right) \, d\ind{\tau\sub{i} < t}. \ey

Combining equations (\ref{nodef}) and~(\ref{def2}) we obtain the
updating process for $p\sub{j}(t)$:
\begin{eqnarray}\label{upd2}
dp\sub{j}(t) &=& p\sub{j}(t) \sum_{i=1}^N \left[ \left(
\frac{\lambda\sub{ij}(t^{-})}{\lb\sub{i}(t^{-})} - 1 \right)  \,
\left( d\ind{\tau\sub{i} \leq t} -\lb\sub{i}(t^{-})\,
\ind{\tau\sub{i} > t} \, dt\right) \right].
\end{eqnarray}
We note that this process has many intuitive properties.  First, if
the prior $p\sub{j}(t) = 1$ for some-$j$, (and thus $p\sub{j'}(t) =
0$ for all other $j'$), then there is no updating. That is, in an
economy where the agents know for sure the intensity of the firms,
then there is no learning to be done. Second, from
equation~(\ref{nodef}), when no default is observed over an interval
$dt$, then investors revise downward the `high-default' states of
nature (i.e., those $j$ with $\lambda\sub{ij}(t^{-}) >
\overline{\lambda}\sub{i}(t^{-})$),  and in turn revise upward the
`low-default' states of nature (i.e., those $j$ with
$\lambda\sub{ij}(t^{-}) < \overline{\lambda}\sub{i}(t^{-})$).
Conversely, as we see from equation~(\ref{def2}) that when a default
is observed over an interval $dt$, investors revise upward those
high-default states of nature, and in turn revise downward those
low-default states of nature. Third, note that $p\sub{j}(t) \equiv
\mbox{E}\left[ \left. \tilde{S} = j \right| {\cal F}\sub{t} \right]$
is a martingale in that $\mbox{E}\sub{t} \left[ dp\sub{j}(t) \right]
= 0$, as can be seen from equations~(\ref{lambar2})
and~(\ref{upd2}). Indeed, it is convenient to rewrite
equation~(\ref{upd2}) as
\begin{eqnarray}\label{upd3}
\frac{dp\sub{j}(t)}{p\sub{j}(t)} &=& \alpha\sub{ij}(t^{-}) \,
dM\sub{i}(t),
\end{eqnarray}
where we have introduced $\alpha\sub{ij}(t^{-}) \equiv \left(
\frac{\lambda\sub{ij}(t^{-})}{\lb\sub{i}(t^{-})} - 1 \right)$ and
the martingale $M\sub{i}(t)$ with dynamics \bq dM\sub{i}(t) \equiv
\left( d\ind{\tau\sub{i}\leq t}-\lb_i(t)\ind{\tau\sub{i}>t} dt
\right). \eq

Finally,  if we make the additional assumption that, for the typical
firm-$i$, the conditional intensities $\lambda\sub{ij}(t)$ are
increasing in $j$: \bq\label{order_lambda} \lambda\sub{i1}(t) <
\lambda\sub{i2}(t) < \ldots < \lambda\sub{iJ}(t) \eq then we can
show that the default intensity $\overline{\lambda}\sub{i}(t)$
increases conditional on a default of some other `typical'
firm-$i'$. Indeed, for the special case where the conditional
intensities $\{ \lambda\sub{ij}(t) \}$ are assumed to be constants,
we can write \by \left. d\overline{\lambda}\sub{i}(t)
\right|_{d\ind{\tau\sub{i'} < t}=1} &=& \sum_{j=1}^{J}
\lambda\sub{ij} \, \left. dp\sub{j}(t)
\right|_{d\ind{\tau\sub{i'} < t}=1} \nn \\
%
&=& \sum_{j=1}^{J} \lambda\sub{ij} \, p\sub{j}(t) \, \left(
\frac{\lambda\sub{i'j}(t)}{\overline{\lambda}\sub{i'}(t)} - 1
\right)
\nn \\
%
&=& \left( \frac{1}{\overline{\lambda}\sub{i'}(t)}\right) \left\{
\mbox{E}\sub{j} \left[ \lambda\sub{i} \lambda\sub{i'}\right] -
\mbox{E}\sub{j} \left[ \lambda\sub{i} \right]  \mbox{E}\sub{j}
\left[ \lambda\sub{i'} \right] \right\}
\nn \\
%
&=& \left( \frac{1}{\overline{\lambda}\sub{i'}(t)}\right)
\mbox{Cov} \left[\lambda\sub{i}, \, \lambda\sub{i'} \right] \nn \\
&>& 0 \qquad \ey for firms $i, i'$ satisfying
equation~(\ref{order_lambda}).  Intuitively, this captures the
notion that, after a default of firm-$i'$, investors attribute a
higher default intensity for most other firms.

The model specified by equations~(\ref{lambar}) and~(\ref{upd2}) is
reminiscent of the counterparty risk example of JY given in
equations~(\ref{JY2a})-(\ref{JY2b}) above in that the intensity of
default $\lb\sub{i}(t)$ for firm-$i$ increases when some other
firm-$k$ defaults. In contrast to JY, however, contagion is
explicitly modeled as a result of the updating of beliefs.  Besides
providing a second mechanism for generating contagion (which seems
to be consistent with the Enron and RJR LBO events), the advantage
of this framework is that it remains tractable even when the number
of firms $N$ that share in the contagion is large.  Indeed, we find
that the survival probability for any firm-$i$ is given by: \by
\E[\ind{\tau\sub{i}>T}|\F_t] &=& \sum_{j=1}^{J} p\sub{j}(t) \,
\E[\ind{\tau\sub{i}>T}| \tilde{S} = j, \, \F_t] \nonumber \\
\label{def33}
%
&=& \sum_{j=1}^{J} p\sub{j}(t) \, \E[e^{-\int_t^T \lambda\sub{ij}(s)
\, ds}| \F_t] \ey We emphasize that the survival probabilities in
equation~(\ref{def33}) are {\bf  not} equal to \bq\label{nocox} \E
\left[ \left. \ind{\tau\sub{i} >T}\, \right| \,\F_t \right] \neq \E
\left[ \left. e^{-\int_t^T \lb(s) \, ds} \, \right| \,\F_t \right]
\equiv \E \left[ \left. e^{-\int_t^T \left( \big \!\! \sum_{j=1}^{J}
p\sub{j}(s) \lambda\sub{ij}(s) \right) \, ds} \, \right| \, \F_t
\right] . \eq This result follows from the fact that the intensity
of default for firm-$i$ jumps at the same date that the default
occurs.  That is, our model falls outside of the `no-jump' framework
of Duffie, Schroeder and Skiadas (DSS 1996), or the Cox-process
framework of Lando (1998), and Duffie and Singleton (DS 1999).
\nocite{lan98} \nocite{dufsch96} \nocite{dufsin99}

Not only does our framework tractably account for a large number of
firms sharing in the contagion, but, as we demonstrate below, our
framework remains tractable even if  the intensities $\{
\lambda\sub{ij}(t) \}$ follow affine or squared-Gaussian stochastic
processes.\footnote{The affine restriction on the
$\{\lambda\sub{ij}(t)\}$ is only necessary to obtain closed form
solutions for prices or survival probabilities. The filtering
equation for the prior $\pH(t)$ holds for arbitrary (positive
integrable) conditional intensity processes.}  For example, a very
tractable framework obtains if the $\lij(t)$ are specified as
\bq\label{xvar} \lij(t)= a\sub{ij} + \left( b\sub{ij} \right)^{\top}
X(t), \eq where the $a\sub{ij}$ are positive constants, $X(t)$ is
the state vector specified to have square-root affine dynamics
(Duffie and Kan (1996)), and the $b\sub{ij}$ are constant positive
vectors. Such a specification guarantees that the
$\{\lambda\sub{ij}\}$ maintain their ordering as in
equation~(\ref{order_lambda}) so long as $(a\sub{i1},b\sub{i1}) <
(a\sub{i2},b\sub{i2})< \ldots < (a\sub{iJ},b\sub{iJ})$. In
particular, survival probabilities and the prices of typical
defaultable claims, such as risky bonds and credit derivatives, can
be obtained in closed-form at little additional cost over and above
the traditional framework that ignores contagion.  This feature is
essential if one wishes to price, for example, CDO's in a model that
captures contagion over a large number of firms.

Although the model presented above is a reduced-form model, it is
easily reconciled with the structural framework following the
intuition of Duffie and Lando (DL 2000). DL show that, in contrast
to a standard structural model (e.g., Merton (1974)), if the
underlying firm value is imperfectly observed by investors, then
from their point of view the default time becomes inaccessible in
that default arrives as a surprise event. Our framework can be
interpreted as an extension of DL's model to multiple firms that
share a common (but unknown) accounting accuracy. We illustrate this
in Appendix A. We consider an economy with multiple firms whose
asset values are imperfectly known to investors, and whose
`accounting quality' is correlated across firms.  Then, the
unexpected default of one firm will trigger an updating of beliefs
by investors about the shared `accounting quality', and hence will
affect the perceived likelihood of default of other firms in the
economy.

Up to this point, the state variable dynamics have been specified
under the historical measure. In the following section we address
the issue of pricing defaultable securities in the presence of
contagion risk and systematic jump risk.  We do this by introducing
a pricing kernel, which allows us to identify risk-neutral dynamics.


\newcommand{\hg}{{\widehat \gamma}}
\newcommand{\HG}{{\widehat \Gamma}}
\newcommand{\lkj}{\lambda\sub{kj}}
\newcommand{\J}{{\cal J}}

\section{Pricing Risky Bonds}

If the number of firms $N$ that are affected by the default of
another firm is sufficiently large so as to be non-diversifiable,
then such default risk will be priced.
%For example, it can be shown \footnote{available upon request} that
%this model can be supported within a general equilibrium framework where,
%even though each individual firm is modeled as `small' in the sense that it does not
%affect current aggregate production, the information conveyed by the default itself is
%sufficient for it to generate a risk-premium. Motivated by this example,
Here, we consider the pricing of defaultable securities based on a
pricing kernel whose dynamics are sufficiently flexible to generate
both jump risk premia and contagion risk premia. In particular, we
 construct the pricing kernel following the same approach as for
  the default process, by assuming that {\bf if} the state were
known  the risk-adjustment would take the standard form:
$$\left.\Lambda\right|\sub{\tilde S=j}\equiv \Lambda^j(t)=e^{-\int_0^t r^j(us) du} \,
\xi^c(t)\xi^j(t)$$ where $r^j(t)$  is has the interpretation of
being the "shadow" short rate that would prevail in state $j$. The
dynamics of  $\xi^c(t)$ and $\xi^j(t)$ are specified as \by
\frac{d\xi^c(t)}{\xi^c(t)}&=& - \theta(t)^\top \, dz(t),
\label{dxic}\\
%
\frac{d\xi^j(t)}{\xi^j(t^-)} &=& \sum_{i=1}^N
\left(\tilde{\gamma}\sub{ij} \, d\ind{\tau\sub{i}\leq t}- \lij(t) \,
\hg\sub{ij} \, \ind{\tau\sub{i}>t} \, dt \right), \label{dxij} \ey
with initial conditions $\xi^c(0)=\xi^j(0)=1\,\,\forall j$. These
processes capture the market price of diffusion risk ($\theta$) and
the jump risk premia ($\gamma\sub{ij}$) that would prevail if the
economy was in  state $j$.

For simplicity, we make the following technical assumptions about
the various market prices of risk (these assumptions are useful for
applying standard results on changes of measure, such as Girsanov's
theorem). \vspace*{3mm}

\noindent {\it {\bf Assumption (A2)} \noindent The vector of market
prices of Brownian motion risk $\theta_t$ is progressively
measurable with respect to $\F^z(t)$ and satisfies the Novikov
condition. Further, the market prices of jump risk
$\tilde{\gamma}\sub{ij}$ are i.i.d. $\F(\tau\sub{i})$-measurable
random variables with density $f\sub{ij}(\gamma)$, mean
$\hg\sub{ij}= \int \gamma\sub{ij} \, f\sub{ij}(\gamma) \,
d\gamma\sub{ij}$ and finite variance (and  which have the same
support).

 The shadow interest rate
process $r^j(t)$ are positive and  progressively measurable with
respect to $\F^z(t)$.} \hspace*{-2mm}

Since investors do not know which state the economy is in, the
actual pricing kernel is a probability weighted average of the
"shadow" kernels:\footnote{We note that the pricing kernel dynamics
specified in equations~(\ref{ker1})-(\ref{dxij}) can be obtained
endogenously within a general equilibrium exchange economy similar
to Lucas (1978).\nocite{luc78} In that economy the output process
would have different dynamics in $J$ states each of which is not
known to the representative investor, who only learns based on the
arrival of the discrete events that affect aggregate output.}
 \bq\label{ker1} \Lambda(t) =\sum_{j=1}^J \pj(t) \Lambda^j(t) \eq

>From It\^o's lemma, its dynamics are given by: \bq\label{dxi}
\frac{d\Lambda(t)}{\Lambda(t^-)} =-r(t)dt -\theta(t)^\top \, dz(t) +
\sum_{i=1}^N \left(\Gamma\sub{\xi,i} \, d\ind{\tau\sub{i}\leq t} -
\lb\sub{i}(t) \, \HG\sub{\xi,i} \, \ind{\tau\sub{i}>t} \, dt
\right), \eq where the short rate is given by: \bq r(t)=\sum_{j=1}^J
\pj(t)\frac{\Lambda^j(t)}{\Lambda(t)}r^j(t) \eq and $\Gamma_{\xi,i}$
specifies the size of the jump in $\xi(t)$, and is defined via:
\bq\label{gamxii} (1 + \tilde{\Gamma}\sub{\xi,i})\lb_i(t^-)=
\sum_{j=1}^J \pj(\tm)\frac{\Lambda^j(t^-)}{
\Lambda(t^-)}(1+\tilde{\gamma}\sub{ij}) \, \lij(t^-) \, . \eq

Interestingly, the short rate is a probability and utility weighted
average of the shadow short rates.  It is easy to see that the short
rate will experience jumps when a credit event occurs. More
specifically, we have:
\begin{lemma}\label{Qint}
If higher risk states have higher utility costs and lower shadow
rates (i.e., if $\lambda_{\cdot,1}\leq\lambda_{\cdot,2}\ldots$ then
$\Lambda_{\cdot,1}\leq\Lambda_{\cdot,2}\ldots$ and $r^1\geq
r^2\ldots$) then interest rates jump downwards upon occurrence of a
credit event.
\end{lemma}
\begin{proof}
Obvious since upon a credit event updating of beliefs will lead to
shift probability from states with intensity below their average
$\lb(t)$ to states with intensity greater than $\lb$.
\end{proof}
This simple insight offers some plausible explanation for a
"flight-to-quality" like reaction of the risk-free short rate upon
occurrence of a credit event. Indeed, it is plausible that states
with higher risk would be characterized by lower productivity/growth
rates (i.e., lower shadow risk-free rates) and higher marginal
utility.

Similarly, note that the expected jump in the pricing kernel
$(1+\tilde{\Gamma}\sub{\xi,i})\lb_i(t^-)$ is basically a probability
and utility weighted average of the jumps in the shadow kernels
 $\{\tilde{\gamma}\sub{ij}\}$.

The parameter $\HG_{\xi,k}$ represents the market price of jump risk
for firm $k$.  Indeed, if $\HG_{\xi,k}=0$, that is, if the pricing
kernel is not affected by individual firm $k$'s default on average,
then the risk-neutral (instantaneous) probability of default is
unchanged, as implied in the following lemma:
\begin{lemma}\label{Qint}
The risk-neutral default intensity for firm $k$ is given by
$\lb\sub{k}^Q(t)=(1+\HG\sub{\xi,k}) \, \lb_k(t)$.
\end{lemma}

\noindent{\bf Proof:} \by \lb_k^{Q}(t) \, \ind{\tau^k > t} \, dt
&\equiv&
\E^Q_t[d\ind{\tau^k\leq t}] \nonumber \\
%
&=&\E^P_t\left[\frac{d \left( \xi(t)\ind{\tau^k\leq t}
\right)}{\xi(t^{-})}\right]
\nonumber \\
%
&=& \E^P_t\left[ d\ind{\tau^k\leq t}+\ind{\tau^k\leq
t^-}\frac{d\xi_t}{\xi_{t^-}}
+\frac{1}{\xi_{t^-}}\Delta\xi_t\Delta\ind{\tau^k\leq t}\right] \nonumber \\
%
&=& \left( 1 + \HG_{\xi,k}(t) \right) \lb_k(t) \, \ind{\tau^k > t}
\, dt, \ey where we use the fact that $\lb_k(t)$ is the
$P$-intensity of $\tau\sub{k}$, and that $\xi$ is
a $P$-martingale.\eproof\\

\noindent Lemma \ref{Qint} emphasizes that, because each firm's
individual default event is priced (i.e., affects the pricing kernel
via equation~(\ref{dxij})), the risk-neutral default intensity is
different than the physical measure intensity. Assuming that the
average jump in the pricing kernel on a default event date is
positive, i.e., $\HG_{\xi,k}>0$, the risk-neutral default intensity
is higher than the physical measure intensity $\lb^{Q}>\lb^P$.  As
noted by Jarrow, Lando and Yu (2001) and Driessen (2002), the
possibility that $\frac{\lb^Q}{\lb^P}>1$ provides a potential
explanation for why credit spreads are higher than observed expected
loss rates at the short end of the credit spread term structure.

In contrast to these papers, our model identifies {\em two} sources
of risk-premia associated with credit spread jumps for a given
firm-$i$, namely: those associated directly with credit-events of
firm-$i$, and those associated with a contagious response by
firm-$i$ due to the credit event of some other firm-$k$.
Importantly, it is this second source which provides a justification
for the first source, since, if a jump-to-default of some firm-$i$
did not generate a market-wide effect, it is difficult to justify
why it should be priced!

For what follows, it is convenient to define \bq\label{qj} \qj(t)
\equiv \pj(t)\frac{\Lambda^j(t)}{\Lambda(t)}\,\,\,\,\,j=1,\ldots J,
\eq and \bq\label{qlij} \lambda\sub{ij}^{Q}(t) \equiv
\lambda\sub{ij}(t) \left( 1 + \hg\sub{ij} \right). \eq From their
definition, the $\{\qj(t)\}$ are positive and sum to unity. As we we
see below the $\qj(t)$ can be interpreted as the risk-neutral
probability of being in state $j$.  Further,  the
$\lambda\sub{ij}^{Q}(t)$ can be interpreted as the risk-neutral
intensity for firm-i conditional on being in state-$j$.

Equations~(\ref{qj}) and~(\ref{qlij}) allow us to write the
risk-neutral probability of survival in an intuitive form:
\begin{proposition}\label{propqsur}
In the presence of priced credit event risk and contagion risk, the
risk-neutral survival probability of firm $k$ is given by:
\bq\label{propqsureq} \E^Q_t\left[\ind{\tau\sub{k}>T}\right] =
\ind{\tau\sub{k}>t} \sum_{j=1}^J \qj(t) \,
\E^{Q}_t\left[e^{-\int_t^T\lambda\sub{kj}^{Q}(s) \,ds}\right], \eq
where, under $Q$, the process $z^{Q}(t)=z^P(t)+\int_0^t \theta_s ds$
is a standard Brownian motion.
\end{proposition}
\noindent \underline{\bf Proof:} See Appendix~\ref{proofap}.\eproof

\vspace*{2mm} \noindent

We note that the right-hand side expectation in Proposition
\ref{propqsur} can be readily calculated if the risk-neutral
processes for the $\{\lambda\sub{ij}^{Q}(t)\}$ are specified as
affine processes.  In this case, the risk-neutral expectation will
be equal  to a weighted average of exponential affine conditional
survival probabilities each corresponding to the survival
probability of a firm with default intensity
$\lambda\sub{kj}^{Q}(t)$.

A generalization of Proposition 2 provides a pricing formula for a
defaultable claim:

\begin{proposition}\label{prop76}
In the presence of priced credit event risk and contagion risk, the
price of a risky defaultable claim issued  by firm  $k$ which pays
$\${\cal X}$ conditional on no-default, and zero otherwise, is given
by:
\begin{eqnarray}
B\sub{k}(t) &\equiv& \E^Q\sub{t}\left[e^{-\int_t^T r(s) \, ds} \,
{\cal X} \, \ind{\tau\sub{k}
> T}\right] \nonumber \\ \label{bkt}
%
&=&\ind{\tau\sub{k} > t} \sum_{j=1}^{J} \qj(t) \,
\E^{Q}\sub{t}\left[e^{-\int_t^T
\left(r(s)+\lambda\sub{kj}^{Q}(s)\right) \, ds} \, {\cal X} \,
\right].
\end{eqnarray}
\end{proposition}
\noindent{\bf Proof:} Similar to that of Proposition~\ref{propqsur}
and thus omitted.\eproof

\vspace*{2mm} \noindent Proposition~\ref{prop76} allows us to derive
the implied risk premium on a claim subject to default risk and
contagion risk. Indeed, it is convenient to define $B\sub{kj}(t)
\equiv \E^{Q}\sub{t}\left[e^{-\int_t^T
\left(r(s)+\lambda\sub{kj}^{Q}(s)\right) \, ds} \,{\cal  X} \,
\right]$.
%via $B_k(t)\equiv \sum_{j=1}^J \qj(t)B\sub{kj}(t)$.
Then, applying It\^o's lemma to equation~(\ref{bkt}) and using
equations~(\ref{qj}), (\ref{pdyn}), (\ref{dxij}) and~(\ref{dxi}), we
obtain: \bq\label{dB} \frac{dB\sub{k}(t)}{B\sub{k}(t^-)} =
\mu\sub{B}(t) \, dt+\sigma\sub{B}(t) \,dz\sub{t} - \sum_{i\neq k }
\tilde{\Gamma}\sub{k,i} \, d\ind{\tau\sub{i}\leq t} - \Gamma\sub{k}
\, d\ind{\tau\sub{k} \leq t}, \eq where \by
\Gamma\sub{k}&=&1\\
%
\tilde{\Gamma}\sub{k,i}&=& 1 - \left( \frac{\sum_{j=1}^J \qj(\tm) \,
\lij(\tm) \, (1+\tilde{\gamma}\sub{ij}) \, B\sub{kj}(\tm)}{B_k(\tm)
\, \overline{\lambda}\sub{i} \, (1+\Gamma\sub{\xi i})} \right). \ey
The relation $\Gamma\sub{k} = 1$ captures the fact that at the jump
date the value of the risky bond drops to zero.
%Further, we claim that $\HG\sub{k,i}\geq 0$, implying that on average the jump-to-default of
%firm $i(\neq k)$ has a negative impact on the value of security $k$.
%This can be seen by noting that
%\begin{eqnarray}
%\widehat\Gamma\sub{k,i} &=& 1- \frac{\sum_{j}\qj\lij^Q B\sub{kj}}{\lb^Q B_k} \nonumber \\
%%
%&\equiv& 1- \frac{ \overline{\lambda\sub{i}^Q \,B\sub{k}}}{\lb^Q B_k} \nonumber \\
%%
%&=& 1- \frac{ \overline{\lambda\sub{i}^Q} \overline{B\sub{k}}
%+ \mbox{Cov} \left[\lambda\sub{i}^Q\,B\sub{k} \right]}{\lb^Q B_k} \nonumber \\
%%
%&=& - \frac{\mbox{Cov} \left[\lambda\sub{i}^Q \,B\sub{k} \right]}{\lb^Q B_k}.
%\end{eqnarray}
%Thus, so long as the $\lij^Q$ and  $B\sub{kj}$ `covary' inversely as a function of state
%$j$, our claim holds in that the second term on the RHS will be less than one. This is indeed the case since,
%since the $\{B\sub{kj}\}$ are a decreasing function of the $\{\lij^{Q}\}$.
%
The diffusion and drift terms in equation~(\ref{dB}) can also be
computed for specific (e.g., affine) dynamics of $X$, but for the
purpose of identifying the contribution of contagion risk for credit
risk-premia, the above representation suffices. Indeed, we find:
\begin{proposition}\label{RP}
For dates $(t < \tau\sub{k})$, the instantaneous premium for credit
risk is given by: \bq
\frac{1}{dt}\E^P\left[\frac{dB_k(t)}{B_k(t)}\right]-r_t=\sigma\sub{B}(t)^\top
\theta(t) +\sum_{i\neq k} \, \lb_i(t) \, \widehat {\Gamma\sub{k,i}
\, \Gamma\sub{\xi,i}}
 \ind{\tau\sub{i} > t}  + \lb_k(t) \, \Gamma\sub{k} \, \HG_{\xi,k}.
%+\sum_{i\neq k}\lb_i(t) \, \HG\sub{k,i} \, \HG_{\xi,i} + \lb_k(t) \, \Gamma\sub{k} \,
%\HG_{\xi,k},
\eq
%where
%\by
%\HG\sub{k,i}&=& 1-\frac{\sum_{j}\qj\lij^Q B\sub{kj}}{\lb_i^Q B_k}\\
%\HG\sub{\xi,i}&=& \frac{\lb\sub{i}^Q}{\lb_i}-1.
%\ey
\end{proposition}

\noindent{\bf Proof:}  By definition of the pricing kernel we have
\bq \E^P\sub{t}\left[\frac{d \left( \big \xi(t) \, B\sub{k}(t)
\right)}{\xi(t^{-}) \, B\sub{k}(t^{-})} \right]= r(t) \, dt. \eq
Applying It\^o's lemma and using equations~(\ref{dxi})
and~(\ref{dB}) we find: \by \E^P\sub{t}\left[\frac{d\left( \big
\xi(t) \, B\sub{k}(t)\right)}{\xi(t^{-}) \, B\sub{k}(t^-)}\right]&=&
\E^P\sub{t}\left[\frac{d\xi(t)}{\xi(t^{-})}+
\frac{dB\sub{k}(t)}{B\sub{k}(t^-)} +
\frac{d<B\sub{k},\xi>\sub{t}}{B\sub{k}(t^-)\xi(t^-)}+
\frac{\Delta B\sub{k}(t) \, \Delta\xi(t)}{B\sub{k}(t^-) \, \xi(t^-)} \right] \\
%
&=& E^P\sub{t}\left[ \frac{dB\sub{k}(t)}{B\sub{k}(t^-)}
-\sigma\sub{B}(t)^\top\theta(t)\, dt - \sum_{i=1\neq
k}^N\Gamma\sub{\xi,i}\Gamma\sub{k,i}d\ind{\tau\sub{i}\leq
t}-\Gamma\sub{\xi,k}\Gamma\sub{k} d\ind{\tau\sub{k}\leq t}\right].
\nonumber \ey The result then follows from the fact that $\lb_i(t)$
is the $P$-intensity of $\tau\sub{i}$.

%and the definitions of $\Gamma\sub{k,i}$, $\Gamma\sub{\xi,i}$ and $\Gamma\sub{k}$.
%We note that in our model we have the surprising equality
%$\widehat{\Gamma_{k,i}\Gamma_{\xi,i}}=\HG_{k,i}\HG_{\xi,i}\,\,\forall k,i\neq k$, which
%indicates that the jump size in the pricing kernel when $i$ defaults is independent from
%the jump size in security $k$ when $i$ defaults.
\eproof

\vspace*{3mm} \noindent Proposition~\ref{RP} implies that in the
presence of systematic jump risk and contagion there are two
additional sources of risk-premia that can affect the instantaneous
credit spread over and above continuous co-variation risk. First, if
firm $k$'s default event affects the pricing kernel (i.e., if
$\HG_{\xi,k}\neq 0$), then the expected common jump in the firm's
price and the pricing kernel will contribute to the spread
($\lb_k(t) \, \Gamma\sub{k} \, \HG_{\xi,k}$). Second, if there is
contagion in the sense that firm-$i$'s default affects the intensity
of firm-$k$'s default ($\Gamma\sub{k,i}\neq 0$),  and if firm $i$'s
default event is systematic ($\HG_{\xi,i}\neq 0$), then the sum of
all these interaction will also contribute to the spread of firm $k$
through the term $\sum_{i\neq k} \lb_i(t) \, \mbox{E}^{Q}\left[
\Gamma\sub{k,i} \, \Gamma\sub{\xi,i} \right] \ind{\tau\sub{i} > t}$.

\subsection{The size of jump premia}\label{jumpsize}

One disadvantage of exogenously specifying the pricing kernel is
that our model does not generate an estimate of the size of the
pricing kernel jump at default events. However, it is reasonable to
assume that the size of the jump of the pricing kernel $\HG\sub{\xi
k}$ is closely linked to the size of the contagious jump $\HG\sub{i
k}$.\footnote{ Jarrow et al (2005) suggest that
$\HG\sub{\xi k} \rightarrow 0$ as $\HG\sub{i k} \rightarrow 0$ for
sufficiently large $N$.  The conditions on $\HG\sub{i k}$
and $N$ imply that jump risk is
diversifiable and that any given firm has a negligible impact on the economy.}

To illustrate this link we focus on a special case where the
aggregate corporate bond market return is instantaneously
mean-variance efficient, implying that it can be used as a proxy for
the pricing kernel. For simplicity, we assume that the magnitude of
contagion jumps are constant and symmetric across all firms,
$\Gamma\sub{\i k} =\Gamma\sub{cont}$, as are all jumps to default,
$\Gamma\sub{k} = \Gamma\sub{def}$. Further, we assume all bonds
have the same initial price, $B\sub{k}(t) = B$, and volatility, $\sigma\sub{k}
= \sigma$.

Under these assumptions, the current value of the aggregate corporate bond
market is
\begin{eqnarray}
M(t) &\equiv& \sum_{k=1}^{N} B\sub{K}(t)= NB. \label{garb87}
\end{eqnarray}
Furthermore, it follows from equation~(\ref{dB}) that
\begin{eqnarray}
dM(t) &\equiv& \sum_{k=1}^{N} dB\sub{K}(t) \nonumber \\
%
&=& \sum_{k=1}^{N} \left[ \sigma \, B \, dz - B \, \Gamma\sub{cont} \sum_{i
\neq k}^{N}
d\ind{\tau\sub{i} < t} - B \, \Gamma\sub{def} d\ind{\tau\sub{k} < t} \right].
\label{garb88}
\end{eqnarray}
Together, equations~(\ref{garb87})-(\ref{garb88}) imply
\begin{eqnarray}
\frac{dM(t)}{M(t)} &=& \frac{1}{N} \sum_{k=1}^{N} \left[ \sigma  \, dz -
\Gamma\sub{cont} \sum_{i \neq k}^{N} d\ind{\tau\sub{i} < t} -  \Gamma\sub{def}
d\ind{\tau\sub{k} < t} \right] \nonumber \\ \label{meanvar22}
%
&=& \sigma  \, dz -  \frac{1}{N} \left( \bbig (N-1) \Gamma\sub{cont} +
\Gamma\sub{def} \right)
\sum_{i=1}^{N} d\ind{\tau\sub{i} < t} ,
\end{eqnarray}
where we have used the identity $\sum_{k=1}^{N}\sum_{i\neq k}^{N} =
\sum_{i=1}^{N}\sum_{k\neq i}^{N}$.

The assumption that the corporate bond market is mean-variance efficient for
pricing corporate debt implies
that, for some constant $\beta$ we can write expected excess returns as
\begin{eqnarray}
\frac{1}{dt}\E^P\left[\frac{dB\sub{k}(t)}{B\sub{k}(t)}\right]-r(t)
&=& \frac{1}{dt}\E^P\left[\beta \, \frac{dM(t)}{M(t)} \,
\frac{dB\sub{k}(t)}{B\sub{k}(t)}\right] \nonumber \\
%
&\equiv& \theta\sub{diff} + \theta\sub{cont} + \theta\sub{def}, \label{mu_bond}
\end{eqnarray}
where the diffusion, contagion, and default components are
\begin{eqnarray}
\theta\sub{diff} &=& \beta \,\sigma^{2}\\ \label{cont87}
%
\theta\sub{cont} &=& \beta \, \frac{\lambda}{N} \left( \big (N-1) \HG\sub{cont}
+ \HG\sub{def} \right)(N-1) \HG\sub{cont}
\\ \label{def87}
%
\theta\sub{def}  &=& \beta \, \frac{\lambda}{N} \left( \big (N-1) \HG\sub{cont}
+
\HG\sub{def} \right) \HG\sub{def}.
\end{eqnarray}
As predicted above, for sufficiently large $N$, the jump-to-default risk premia
is linear in the size of the
contagion jump $ \HG\sub{cont}$:\footnote{From equation~(\ref{meanvar22}), we
see that under the assumption of mean-variance
efficiency $\HG\sub{\xi i} = \HG\sub{cont}$.}
\begin{eqnarray}
\left. \theta\sub{def} \right|\sub{N \rightarrow \infty}  &\sim& \beta \,
\lambda \,
\HG\sub{cont} \, \HG\sub{def},\label{thetadef}
\end{eqnarray}
implying that jump-to-default risk is priced due to the existence of contagion
risk.
Furthermore, note that, while
the jump-to-default premium is independent of the number $N$ of firms that
share in the contagion,
the contagion premium increases linearly with $N$:
\begin{eqnarray}
\left. \theta\sub{cont} \right|\sub{N \rightarrow \infty}  &\sim & \beta \,
\lambda \,
N \, \HG\sub{cont}^{2}.\label{thetacon}
\end{eqnarray}
Combining equations~(\ref{thetadef}) and (\ref{thetacon}), we obtain the ratio
of the
two premia:
 \bq \left.\frac{\theta\sub{def}}{\theta\sub{cont}}\right|\sub{N \rightarrow
\infty} \sim \frac{\HG\sub{def}}{N \, \HG\sub{cont}}, \eq Contagion
risk is likely to be much larger than jump-to-default risk, given
the estimates we obtain for $\HG\sub{cont}$ in our empirical
investigation below.

We can also consider the relative impact of contagion risk by
noting the parameter $\beta$ is not a free parameter,
but rather can be determined from the expected return of the market
portfolio. Indeed, by pre-multiplying equation~(\ref{mu_bond}) by
$\frac{1}{N} \sum_{k=1}^{N}$ we find
\begin{eqnarray}
\frac{1}{dt}\E^P\left[\frac{dM(t)}{M(t)}\right]-r(t)
&=& \frac{1}{dt}\E^P\left[\beta \, \left(\frac{dM(t)}{M(t)}\right)^{2}\right]
\nonumber \\
%
&=& \beta \, \left[ \sigma^{2} +  \frac{\lambda}{N} \left( \left( N-1 \right)
\Gamma\sub{cont} + \Gamma\sub{def} \right)^{2} \right].
\end{eqnarray}
Since all bonds are initially equivalent, they all have (and the
bond market portfolio has) the same expected return.  Thus, defining
$\mu \equiv\frac{1}{dt}\E^P\left[\frac{dM(t)}{M(t)}\right] =
\frac{1}{dt}\E^P\left[\frac{dB\sub{k}(t)}{B\sub{k}(t)}\right]$, we
find \bq \beta = \frac{\mu - r}{\sigma^{2} +  \frac{\lambda}{N}
\left( \left( N-1 \right) \Gamma\sub{cont} + \Gamma\sub{def}
\right)^{2}}.\label{bigbeta} \eq That is, $\beta$ is the ratio of
the excess return to the variance of the bond market portfolio.
Given reasonable values for the other parameters in this equation,
we can estimate the relative sizes of $\HG\sub{cont}$ and
$\HG\sub{def}$.  As we shall see in the calibration section, the
former is likely to be quite a bit larger than the latter.

Our model focuses attention on the potential for contagion risk to account for
part of the
credit spread puzzle.  This risk can only be an important priced factor for
corporate bonds
if jumps to default actually impact the prices of other corporate bonds.
While we can posit an information channel or a
counterparty channel to explain why this would occur, whether or not such jumps
impact the bond
market to any noticeable extent is an empirical question.  We investigate this
question in
the next section.

%%%%%%%%%%%%%%%%%%%%%%%%%%%%%%%%%%%%%%%%%%%%%%%%%%%%%%%%%%%%%%%%%%%%
\section{Empirical Analysis}
%%%%%%%%%%%%%%%%%%%%%%%%%%%%%%%%%%%%%%%%%%%%%%%%%%%%%%%%%%%%%%%%%%%%

If credit events are priced, we expect the main source of risk to
stem from events that are
surprises to investors.  Unexpected
credit events, then, should be identifiable as those where an
individual firm's bond exhibits a large jump in its spread. In
this section we investigate empirically the impact that such credit
events have on the market. In particular, if these events
are priced, then we expect them to be associated with
increases in aggregate credit spreads, and possibly also have
a negative impact on equities. Furthermore, consistent
with the implications of the general equilibrium framework in
Appendix~\ref{GEap}, we investigate whether such jumps are
associated with `flights-to-quality' affecting risk-free
rates.\footnote{Note that we do not investigate incidences of major
spread decreases, since the effect of these jumps on the `market
portfolio' is not expected to be symmetric.}

Regardless of whether the contagion is due to `counterparty risk' or to
`updating-of-beliefs', one would expect that jumps in the yield spreads of
larger,
`safer' firms would produce a greater impact on the market portfolio than would
`riskier' firms for whom a default is somewhat expected.
As such, we limit our empirical investigation to investment-grade bonds.  As we
noted earlier, few investment-grade firms default while holding the rating.
Thus, if these firms do eventually default, that event is likely to be
anticlimatic and would not likely cause a market-wide response. Consequently,
our credit events consist largely of jumps in spreads, not defaults.

\subsection{Data}
In order to gather a sufficiently large number of credit events in
the investment-grade market, we use the Fixed Income Database (FID),
which contains month-end trader quotes from January 1973-March 1998.
These data have the advantage of reasonably good quality price data
(see Warga (1991) and Warga and Welch (1993)) over a long time series.

We calculate corporate bond spreads as the difference between the
bond's yield to maturity (YTM) and the interpolated YTM on a
Treasury bond with a similar maturity. The Treasury yields are the
Federal Reserve's Constant Maturity Treasury (CMT) daily series, using
only yields from this series in time periods when the bond is actually
auctioned. We use actual
CMT yields when the corporate bonds have the same maturity as the CMT bonds
and interpolated YTMs (using the Nelson-Siegel (1987) method) for the
other maturities.\footnote{Nelson-Siegel interpolation requires estimates
of the rate of decay of the regressors, $\tau$.  We use 400 values of $\tau$ to
obtain the most efficient
estimate of the yield curve for each day in the sample. We use $\tau$ to create
YTMs for each of 30 yearly maturities.}

We use trader quotes and delete matrix
prices from the analysis, as suggested by Warga (1991). We consider spreads on
all U.S. corporate
bonds rated investment-grade in the FID as long as they are not
private placements, medium term notes, or Euro-bonds. We delete
offerings by government-sponsored enterprises and supranational
organizations, as well as mortgage-backed or other asset-backed
securities and bonds that are convertible into preferred stock.

We define a credit event as a major change in a bond's credit
spread from one month to the next. Among the set of bonds that experience such
shocks,
we exclude bonds with less than two years until
maturity.\footnote{We do so because a large jump in the YTM does not
necessarily imply a large negative return on very short term bonds.}
We also exclude bonds that have already
fallen below a flat price of \$80 (which would be the second piece
of bad news about the firm). Furthermore, we exclude bonds where the
post-credit-shock price is above \$95. This exclusion
helps avoid identifying a coding error as a
credit-event.\footnote{A bond might be selling at a sizeable discount (e.g.,
below \$80) simply because
yields have risen substantially since the bond was issued. Likewise,
a bond may sell at a sizeable premium because yields have fallen since
issuance,
so that even if this bond suffers a credit event its
price might remain above \$95.  These
types of errors would only reduce the likelihood of finding significant
results.}

The FID does not readily identify floating rate bonds. Unnoticed,
these bonds' spreads could be miscalculated, giving rise to spurious
credit events when none occurred. We eliminate floating rate bonds
from the sample by investigating their description in the Securities
Data Corporation database and by checking that the standard
deviation of a bond's coupon in the FID is zero.

With these qualifications, we obtain 52,828 instances where spreads
widen. Table 1 shows the distribution of spread increases
on corporate bonds in the FID over the sample period. The vast majority of the
increases are
quite small. Indeed, less than 3\% are more than 50 bp. We wish to focus on
rare events, yet
we also wish to avoid small sample problems.
Thus, we consider all spread increases of 200 bp or more (totalling 158 bonds)
as credit events.

To avoid spurious results, we investigate each of these large spread
changes in the financial press (Lexis-Nexus and
Standard and Poor's Creditweek) to ascertain that a credit event
actually occurred.  Evidence of an event includes news of a bond rating
downgrade, dividend
cuts, major losses or other negative information in an earnings
announcement, depressed stock prices, a major lawsuit or accident, a
subsequent default or bankruptcy, or a leverage-increasing merger.
If we cannot find evidence of a credit shock to the bond, we also
check the bond price in Moody's Bond Record to see if a
sale price exists at a level close to our bond price. If not, we assume it
is a typo in the FID. Of the 158 bonds with wider spreads, we have evidence
that 112
involve a credit event. These bonds belong to 40 firms, two of
which suffered two episodes of credit risk (Chrysler and RJR
Nabisco). The large number of bonds relative to the number of firms
reflects the fact that many of the firms have numerous bonds
outstanding.

Most of the credit events involve economic hardship. Many of these bonds belong
to Chrysler, which
suffered in the 1970's and again in 1990. Some of these occur during recession
years.
Another common event is a leveraged buyout (e.g., RJR Nabisco). The third
largest category includes banks that lent
funds that are unlikely to be recovered, often because of real estate loans in
the 1990-1991 recession
or loans to Latin America in the early 1980s.

Some of the corporate bonds in our sample lost substantial value in
one month, and then went on to lose additional value in ensuing
months. Because it is conceivable that such bonds were no longer
considered `investment-grade' in the minds of the marketplace after
the first event, we use only the first credit shock in a series of
episodes. By a series of episodes, we mean more than one credit
shock for a bond over the course of a year.  We identify 25 months
over the sample period in which a credit event first occurs. The remaining
273 months are not credit event months.

We compare these 25 credit event months with other months by
examining how returns of aggregate portfolios differ when a credit event
occurs. We
investigate the impact on the Lehman corporate bond index, the CRSP
value-weighted stock index
(including all NYSE, AMEX and Nasdaq firms), and the Lehman
Treasury index.  Our initial analysis compares credit event months with other
months
using t-tests. Subsequent results
control for other factors in regressions, including controlling for
macroeconomic factors identified by Fleming and Remolona (1999)\nocite{flere99}
and other
episodes of flight-to-quality (see Longstaff (2004)).
We expect that credit events of large firms will have a greater impact on the
market than those of small firms. We measure size by the amount of corporate
bonds outstanding and by
total assets.\footnote{Corporate bonds outstanding is the sum of bonds in the
FID
for the issuer's six-digit cusip, which can be inaccurate if cusips vary across
a firm's bond issues.
Total assets is for the year in which the
credit event occurs and comes from Compustat, except for one
firm whose assets are from Moody's Transportation Manual.}

A major concern is whether our bonds constitute such a
large fraction of the corporate bond index that their own price
movements drive the returns on the Lehman index. This is unlikely as our events
usually involve only a single firm in any given month. The month with the
largest number of affected bonds is
September 1990, during which 11 firms and 27 bonds are classified
as experiencing a credit event. These do not drive the index, as
3811 corporate bonds are in the index that month.
Furthermore, we find similar results (unreported) if we include only
those months where a single firm suffered a credit shock.

\subsection{Results}

We first compare average returns in the Treasury, corporate bond and
stock markets in the months in which credit events occur to the
average returns during those months when no event occurred. We focus on
returns to the corporate bond index in excess of Treasury returns.
Table 2 shows that in the months in
which a credit shock occurred the average excess return on the
corporate bond index is negative and significantly lower than the
return in the other 273 months (-0.33\% in months with a credit
event and +0.06\% in other months). As expected, the difference in
excess returns is mainly driven by the largest firms: measured by
bonds outstanding, these firms' events lead to an average excess
corporate bond return of -1.05\%; among firms
with the greatest assets, the average drop was 53 bp. These negative
excess returns often occur with flights to quality or other times
when Treasury returns are unusually large. Returns on stocks are
unaffected by these credit shocks on average.  This would be
expected if a large fraction of the credit events stem from
leveraged buyouts and other events that are positive news for
shareholders.


What if causality runs from Treasuries to corporate bonds in
Table 2? If Federal Reserve easing due to a weak economy causes high
Treasury returns just at the time when perceived credit risk is increasing,
we might see similar market returns.  If
so, then a large number of credit spread increases (reflecting the
weak economy) would randomly result in at least one corporate bond
suffering a 200bp spread increase. First, we note this is highly unlikely given
that our event months are identified both by a large spread increase and news
in the financial press
indicating a reason for the jump.  However,  we can also test this story by
investigating the 273 months that are not associated with a credit event to see
if highest returns in the Treasury market often
cause higher spreads. Specifically, we examine the 25 months with the best
Treasury returns and test the  excess return for the corporate bond market
during
those months. We find they actually have significantly positive excess returns
on corporates, indicating again that Treasury returns do not cause the observed
corporate bond events.

\subsection{Regression Analysis}\label{regression}

The t-tests in Table 2 implicitly assume that only credit shocks
affect monthly asset returns, as other factors are ignored. In
Table 3 we report regressions that control for other effects,
such as macroeconomic factors. In the corporate bond and stock
market regressions we also include the slope of the term structure,
as Estrella and Hardouvelis (1991) show that it predicts recessions.
For the Treasury return regression, we do not include the
slope of the Treasury curve as an explanatory variable, as it may
cause a spurious relationship in the estimation. To control for the
effect of changes in real rates, we also include the current month's
change in the actual Federal Funds rate or alternatively the Federal
Funds rate relative to inflation (the Taylor rule variable:
FF-inflation-2 percent (see Taylor (1993)).

We also include measures of flight to quality (FTQ) in the Treasury
market: an indicator variable for instances of FTQ
that are not related to the credit events we identify and
changes in institutional money market mutual fund flows.
Longstaff argues that changes in money fund flows are indicative of changes
in sentiment that occur with a FTQ.  However, money funds were in their infancy
at the start of our sample period, leading to very high growth rates that
undoubtedly
do not reflect such changes in sentiment. Nevertheless, we investigate the
effects
for comparison sake.

Our indicator variable for the other FTQ episodes is
based on information culled from the financial press. For the part
of the sample from June 1979 on, we set the indicator to one
whenever the Wall Street Journal (WSJ) uses the phrase ``flight to
quality" to explain why prices of Treasuries have risen in that
calendar month.  If the WSJ uses the phrase to describe a long term
trend (say 3 months) in Treasury prices, we do not count it as FTQ.
For the period from 1973 to May 1979, the WSJ index is not
available in electronic form and we search for references to FTQ in three
sources:
(1) the weekly write-up of the bond market in Moody's Bond Survey; ((2) market
commentary of Aubrey Lanston (various issues, 1973-1979), a
government bond trading firm; and (3)Lexis-Nexus.
%Cite for Aubrey Lanston is Lanston, Aubrey G. "Comparative Yields of U.S.
Treasury securities"
%(weekly market commentary).
If any of the sources mention FTQ in Treasuries
comparable to those used for the WSJ, we set the indicator to one
for that month.

We obtain closing VIX implied volatilities from the CBOE website
(data begins in 1986). Corporate bond upgrade to
downgrade ratios are obtained from Moody's Investors Service, as are
monthly default rates.  Because the VIX severely limits the time
span of analysis, we proxy for the VIX with historical stock return
volatility (volatility is calculated using the variance of daily
returns on the S\&P 500, including dividends).

The effects of credit events on corporate bonds in Table 3 are
similar to those found in the t-tests, again with a larger effect
occurring when the credit event is associated with bigger firms.
(These regressions are corrected for heteroskedasticity and
autocorrelation by the Newey-West method using five lags).  In the
first column of Table 3, which uses an indicator for all credit
shocks regardless of firm size, excess corporate returns are
significantly lower in credit event months. Meanwhile,
the signs of the control variables are largely as expected:  a steep
yield curve implies a strong economy and thus a lower chance of
default, while Fed tightening indicates that the peak in the current
business cycle has arrived (with defaults increasing soon). The
trade deficit situation is also very significant, which may reflect
the fact that the corporate bond market includes more exporters than
the economy in general. Defaults and rating changes are not
significant, regardless of the lead/lag relationship used in the
regression (results not reported).

Several of the macro variables are insignificant, and are dropped
for the sake of parsimony in the remaining specifications.
Likewise the FTQ indicator is not important for
the corporate bond market. (In results not reported we also find
that using the on the run/off the run spread as a proxy for FTQ is not
significant.)
This likely reflects the relative safety of bonds in these other episodes that
are
not related to credit risk in the bond market.

In the next two columns of Table 3 we report regressions of excess
bond returns where the credit event months are split into those with
large firms and those with small firm shocks (size is measured by
firm assets). As in the t-tests, the impact is greatest in months
when large firms experience credit shocks. These two columns differ
in the treatment of stock return volatility.  One includes the
VIX volatility measure, which is only available from 1986. The VIX affects
the coefficient on the indicator for the October 1987 crash  as well as
the default rate and ratio of upgrades to downgrades, but this largely reflects
the time period for which the VIX is available.  We further
consider the role of volatility by replacing the VIX with the monthly
standard deviation of daily S\&P 500 returns (including dividends).
This variable has no discernible impact on corporate bond
returns.  We also include institutional money fund flow innovations in the
second
model, but its coefficient is not significant.


The two middle columns of Table 3 show the results of regressions
explaining Treasury returns.  The corporate credit shocks
have a positive impact on Treasury returns, suggestive of a flight
to quality effect. The coefficient on the FTQ indicator implies
this phenomenon occurs in many more situations than just credit events. While
the credit
shock indicator has a coefficient of 54 bp, the other instances of FTQ
have an average impact of more than 70 bp.  The October 1987 stock market crash
likely has a much larger FTQ impact. The model in the second
Treasury specification indicates that credit shocks mainly impact the
Treasury market when they involve large firms.\footnote{In results not
reported, the on
the run/off the run spread and the money fund variable are not significant.}

The last two columns of Table 3 shows regressions analyzing the
effect of our credit shocks on excess stock returns.  Unlike the t-tests, these
estimations indicate that CRSP stocks suffer in months with credit
events. These regressions control for macroeconomic factors, such as
the Federal Funds rate and consumer confidence.  For stocks, higher consumer
confidence always leads to higher returns. This variable impinges on
the significance of the credit events, suggesting that confidence
drops when these events occur. In the second specification, large firm
credit event months are significantly negative once we drop the
consumer confidence variable.

Our results suggest that unexpected credit events
lead to significant losses on a
corporate bond portfolio.  This is consistent with our model's assumption that
credit events affect the pricing kernel.  In the next section, we consider the
impact of this contagion risk on ex ante returns.

%%%%%%%%%%%%%%%%%%%%%%%%%%%%%%%%%%%%%%%%%%%%%%%%%%%%%%%%%%%%%%%%%%%%
\section{Calibration of the model}
%%%%%%%%%%%%%%%%%%%%%%%%%%%%%%%%%%%%%%%%%%%%%%%%%%%%%%%%%%%%%%%%%%%%

In this section we calibrate the model to obtain estimates of the impact of
contagion risk on bond returns. We can approach this exercise from two angles:
(1) we can use estimates of $\beta$ and equation~(\ref{bigbeta}) to get at
$\Gamma\sub{cont}$ and therefore the risk premium, or (2) we can use estimates
of $\Gamma\sub{k,i}$ from the time series regressions in
section~(\ref{regression}) and information about the variance of the pricing
kernel to get at the premium. For completeness we use both methods.

However we approach the calibration, we require estimates of the number of
firms at risk (N), the probability of a credit event $\lambda^{P}$, and the
loss on the bond that jumps to default($\Gamma\sub{k}$). One can estimate
$\lambda^{P}$ for
investment-grade debt in several ways. For example, we could use the one-year
default rate for investment-grade debt, or $\lambda^{P} \approx 10^{-2}$.
However, this
number is too high given that most bonds that default within a
year of having an investment-grade rating fall to speculative-grade before
defaulting.
Alternatively, we can estimate the frequency of credit events from our sample
data.
Given 264,000 observations in the sample, Table 1 implies that the intensity of
a credit event is on the order of $10^{-3}$.
As for N, we calculate that between 1985 and 2001 the number
of firms with investment-grade credit ratings on Compustat ranged from just
over 1000 to just
under 1500, with the higher figures occurring in later
years. We set $\Gamma\sub{k}=50\%$, which is
approximately the historical average recovery rate reported by Moody's (see
Keenan, Shtogrin, and
Sobehart (1999)).\nocite{keesht99} However, more recent defaults,
suggest an even larger loss given default(see Hamilton, Varma, Ou and Cantor
(2003)). Note this parameter has an upper bound of one due to limited
liability.

Our first approach also relies on estimates of Sharpe ratios, $\beta$ and
diffusion volatility.  Given the assumptions in section~(\ref{jumpsize}), we
can use the corporate bond Sharpe ratio and excess returns to in the bond
market to estimate $\beta$.  This parameter value and a measure of diffusion
volatility allow us to get at the components of bond market volatility that
reflect jump risk.  In particular, we obtain $\Gamma\sub{cont}$ from:

\[\beta = \frac{\mu - r}{\sigma^{2} +  \frac{\lambda}{N}
\left( \left( N-1 \right) \Gamma\sub{cont} + \Gamma\sub{def}
\right)^{2}}\]

\noindent This allows us to decompose bond returns and find the premium due to
contagion risk:





\end{document}


.... Should point out that obtain an upper bound on
$\lambda^Q/\lambda^P$ when attribute all contagion jumps to default
events. Instead, many of the events are just jumps in spreads in
response to new information and thus contribute to higher intensity
for the jump in the default intensity and not to the JTD per se.
Perhaps we should distinguish, Credit Event from JTD
risk-premium???? }


This result contradicts the empirical literature on reduced-form
models, which attributes a much larger share of the excess return on
corporate bonds to jump-to-default risk premium than our analysis
suggest is consistent with equilibrium (and volatility bounds on the
pricing kernel). Instead, our analysis suggests that jump to default
premium  must be quite small and therefore that reduced-form models
should investigate other sources of risk, such as liquidity or
catastrophe risk or at least impose some upper bound on its size
during empirical exercises.


We first explain the source of Jump-to-default risk-premia in
reduced-form models and show that they are tied to jumps in the
pricing kernel. We then introduce a simple equilibrium model where
price contagion occurs on event dates. We derive expected excess
returns in that model and decompose it in diffusion risk, JTD risk,
and contagion risk-premium. In the third section we investigate
empirically whether price contagion is present in the data. Finally,
we calibrate our equilibrium model and discuss implied magnitude of
JTD and contagion risk-premia. In an appendix we present a model
consistent with our main empirical results, where price `contagion'
occurs as a result of rational learning.


ADD PART ABOUT DS AND DSS, TO OBTAIN SIMPLE SOLUTIONS, SPECIFY MODELS WHERE THERE
IS NO CONTAGION.

ALSO ADD THAT WE FIND NON-NEGLIGIBLE LAMBDA-Q/LAMBDA-P, BUT THAT THIS DOES NOT LEAD TO
MUCH CREDIT SPREAD/ RISK PREMIA.  INSTEAD, MOST OF THE RISK PREMIA IS DUE TO CONTAGION RISK.

EMPHASIZE THAT ONE YEAR DEFAULT RATES GREATLY EXAGGERATE LAMBDA-P.  INSTEAD, MOST INVESTMENT
GRADE DEBT THAT DEFAULTS WITHIN A YEAR HAS SEVERAL 'DOWNGRADES' (WHETHER LITERALLY OR JUST
EFFECTIVELY IN THEIR PRICE), IMPLYING THOSE WHO FOLLOW A STRATEGY OF SELLING NON-INVESTMENT
GRADE DEBT FACE VERY LITTLE JUMP TO DEFAULT RISK.
