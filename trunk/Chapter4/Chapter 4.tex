\documentstyle{article}
\oddsidemargin=.15in
\evensidemargin=.15in
\textwidth=6in
\topmargin=-.5in
\textheight=9in
\parindent=0in
\pagestyle{empty}

\begin{document}
\section*{Chapter 4}
Representative Consumer with pa... given by Utility Function $U(c, l)$ with properties
\begin{list}{ }
 \item 1) $U_c > 0, U_l > 0$ (more preferred to less)
 \item 2) if $U(c_1, l_1) = U(c_2, l_2)$ then $U(\alpha c_1 + (1-\alpha) c_2, \alpha l_1 + (1-\alpha) l_2) > U(c_1, l_1)$
 \item 3) $(c, l)$ are ``normal goods'' $\frac{\partial c(I, w)}{\partial I} \vert_w > 0$ \\
 \end{list} 
Consider ISO-Utility curve, \\ $U(c(l), l) = const$ 
\begin{eqnarray}
 dU &=& 0 \nonumber \\
 &=& U(c + \Delta c, l + \Delta l) dl\nonumber\\
 -\frac{\partial c}{\partial l} \vert_u &=& \frac{U_l}{U_c}\\
 2^{nd} ...: U(c, l) &=& U(c + \Delta c, l + \Delta l), with \Delta c \equiv c(l+\Delta l) - c(l) \approx \Delta l c_l + \frac{1}{2} \Delta l^2 c_{ll}\nonumber\\
 0 &=& \Delta c U_c + \Delta l U_l + \frac{1}{2} \Delta c^2 U_{cc} + \frac{1}{2} \Delta l^2 U_{ll} + \Delta c \Delta l U_{cl}\nonumber\\
 &=& (\Delta l c_l + \frac{1}{2} \Delta l^2 c_{ll})U_c + \Delta u_l + \frac{1}{2} U_{cc} \left[\Delta l^2 c^2_{ll}\right] + \frac{1}{2} \Delta l^2 U_{ll} + \Delta l^2 c_l U_{ll}\nonumber\\
 \Delta l^1 : 0 &=& c_l U_c + U_l\nonumber\\
 -C_l &=& U_l/U_c\\
 \Delta l^2 : 0 &=& \frac{1}{2}c_{ll} U_c + \frac{1}{2} c^2_l U_{ll} + \frac{1}{2} U_{ll}+ c_l U_{cl}\nonumber\\
 -\frac{1}{2}U_c C_{ll} &=& \frac{1}{2} U_{cc} [\frac{U_l}{U_c}]^2 + \frac{1}{2} U_{ll} - \frac{U_l U_cl}{U_c}\nonumber\\ 
 C_{ll} &=& - \left( \frac{U_l^2}{U_c} \right) \left\{\frac{U_{cc}}{U_c^2} + \frac{U_{ll}}{U_c^2} - 2 \frac{U_{cl}}{U_c U_l}\right\} \nonumber
\end{eqnarray}
Note that $(2)$ above implies for those cases where 
\begin{eqnarray*}
U\left[c + \Delta c, l + \Delta l \right] &=& U \left[ c-\Delta c, l - \Delta l \right]\\
\Delta C U_c + \Delta l U_l &=& 0\\
-\frac{\Delta C}{\Delta l} \vert_U &=& \frac{U_l}{U_c}\\
\end{eqnarray*}
We have,
\begin{eqnarray}
U(c,l) &>& \frac{1}{2} \left\{ U(c+\Delta c, l+\Delta l) + U(c-\Delta c, l- \Delta l) \right\} \nonumber\\
0 &>& (\Delta c U_c + \Delta l U_l + \frac{1}{2} \Delta c^2 U_{cc} + \frac{1}{2} \Delta l^2 U_{ll} + \Delta c \Delta l U_{ll}) + \nonumber\\
&& (-\Delta c U_c - \Delta l U_l + \frac{1}{2} (-\Delta c)^2 U_{cc} + \frac{1}{2} (-\Delta l)^2 ...) \nonumber\\
&>& \Delta c^2 U_{cc} + \Delta l^2 U_{ll} + 2 \Delta c \Delta l U_cl\nonumber\\
&>& \Delta l^2 \left[ \left( \frac{U_l}{U_c}\right) ^2 U_{cc} + U_{ll} + 2 \left( -\frac{U_l}{U_c}\right) U_{cl}\right]\nonumber\\
0 &>& \frac{U_{cc}}{U_c^2} + \frac{U_{ll}}{U_l^2} - 2 \frac{U_{cl}}{U_c U_l}\\
\mbox{ and thus } C_{ll} &\equiv & \frac{\partial^2 C}{\partial l^2} \vert_u > 0
\end{eqnarray}

Indifference ... have two properties :
\begin{list}{ }{}
\item 1) Slopes downward : $ \frac{\partial c}{\partial l} \vert_U < 0 $
\item 2) Convex : $ \frac{\partial^2 c}{\partial l^2} \vert_U > 0 $
\end{list}
Def: $MRS_{L,C}$ of leisure for consumption = - slope of indifference curve = $- \frac{\Delta c}{\Delta l}$ \\
Convex indifference curves imply diminishing MRS = 0
\subsection*{Budget Constraint}
\begin{list}{-}{}
\item Assume agent is a price taker
\item Assume no money. i.e., we have a ``barter economy''
\item Agent endorsed with $h$ units of labor \\ $l + N^s = h = $... plus labor supplied \\
\end{list}
Def : $ w \equiv wage = \frac{\# Consumption ...}{... labor}$\\
\begin{list}{-}{}
\item Use consumption good as ``...''
\item Wage income = $ w N^s$
\item Dividends minus Tax = D - T \\ here T = lumpsum tax = tax independent of actions of agent\\
\end{list}
``Taxes that are not mump sum have important effects on prices, which in turn affects the demand of that good'' - ``Distorting Effect'' \\ \\
Recall Diposable income = $(D-T) + w N^s$ equals consumption C since one period, no savings motive.\\
$ C = w (h - l) + D - T = $ ``budget constraint'' \\ \\
In slope intercept form, $C = (w^(h+D-T)) - wl$ since C is num... rates why we put it on y-axis $-\frac{\partial c}{\partial l}$\\ \\
Optimal consumption bundle = $(C^*, l^*)$ on highest indifference curve consistent with budget constraint.
\begin{list}{-}{}
\item Indifference curver tangent to budget is constant
\item Thus, at optimal $MRS  = -\frac{\partial c}{\partial l} = \frac{U_l}{U_c} = wage$
\end{list}
Recall $wage \equiv $``Price of one unit of leisure, measured in consumption units'' \\ \\
Assumption of constrained optimality provides predictions about one consumer corresponds to changes in 
\begin{list}{ }{}
\item (i) Budget Constraint
\item (ii) Wage Plans
\item (iii) Performance
\end{list}
\textbf{\underline{(i) Pure Income Effect: }} 
\begin{list}{ }{}
\item Change $(D-T)$, holding $w = constant $
\item Budget constraint ... parallel fashion ... slope = wage = constant
\item Due to assumption that $(c,l)$ are normal ..., both increase with an increase in dividends/... in taxes.
\end{list}
\textbf{\underline{(ii) Change in wages holding $D - T$ constant: }} Although typically supply of a good increases in its prices, not obvious for the labor, due to counter-acting income + substitution effects.\\


\textbf{\underline{Substitution Effect:}} At new wage (i.e., wage) ... divided to find target or original indifference curver. New optimal point will be lower leisure/high consumption to effect the higher cost of leisure.\\

\textbf{\underline{Income Effect:}} Add ... dividend $ \Longrightarrow $ will incease both (consumption and leisure) ... consumption increases, leisure not sure. Assumes substitution effect dominates income effect.\\

When plotted (wage vs labor), should understand that more fundamental is the constrained utility ... for a given wage and then consider multiple wage possibiliites. Some with changes in $(D-T)$.\\

\subsection*{Representative Firm}
Assume firm owns its capital (instead of costing it)\\

Production Function: $ Y = Z F(K, N^d)$\\
$Z = $ Total factor productivity\\
$Y = $ Output\\
$K = $ Quantity of Capital\\
$N^D = $ Quantity of labor\\

Here, we consider only a 1- period modeling with K exogin... and $N^D$ a ``control''\\

Defn: Marginal Product of Labor (MPL) = $ \frac{\partial Y}{\partial N_D} \vert_K$ \\
Marginal Product Captial (MPK) = $ \frac{\partial Y}{\partial K} \vert_{N_D}$ \\

Five Key Properties of Prodcution Function:\\
\begin{list}{ }{}
\item (1) Constant Returns to Scale : $F(\lambda K, \lambda N^d) = \lambda F(K, N^d)$
\item (2) $ \frac{\partial Y}{\partial K} \vert_{N^d} > 0, \frac{\partial Y}{\partial N^d} \vert_{K} > 0 $
\item (3) $ \frac{\partial^2 Y}{\partial K^2} < 0 $
\item (4) $ \frac{\partial^2 Y}{\partial N^2} < 0 $
\item (5) $ \frac{\partial^2 Y}{\partial N \partial K} > 0 $
\end{list}

Example Cobb Douglass : $Y = Z K^{\alpha} N^{1-\alpha}$\\
\begin{eqnarray*}
Profit : D = Y &=& Z K^{\alpha} N^{1-\alpha} - wN - RK\\
FOC : \frac{\partial}{\partial N} = 0 &=& (1- \alpha) Z K^\alpha N^{-\alpha} - w\\
(1-\alpha) \frac{Y}{N} &=& w\\
FOC : \frac{\partial}{\partial K} = 0 &=& \alpha Z K^{\alpha -1} N^{1-\alpha} - R\\
\alpha \frac{Y}{K} &=& R
\end{eqnarray*}

Note that these two FOC's imply that $RK + wN = Y$, which further implies that profit $D = 0$. If this were not the case, then profits could be scaled up arbitarily high due to constant returns to scale.\\

$\left. 
\begin{array}{ll}
\mbox{Factor $(1-\alpha) Y \equiv (w, N)$ goes to labor,}\\
\mbox{Factor $\alpha Y = (R,K)$ goes to capital.}
\end{array} 
\right\}\mbox{Emprically $\alpha \approx 0.36$} $ \\

Note, $(Y, K, N)$ can be measure, but not Z.\\

This leads to concept of ``Slow Residual''\\
$Z_* = \frac{Y_*}{K_*^{0.36} N_*^{0.64}}$\\

\textbf{\underline{Profit Maximization : }} Assume f... is wage taking  and owns capital\\
\begin{eqnarray*}
D &=& ZF(K, N) - wN \\
\mbox{FOC : } 0 &=& Z \frac{\partial F}{\partial N}\vert_K - w\\
wage &=& \mbox{ Marginal Product of labor}
\end{eqnarray*}

Thus, the dowmward sloping convex relation before MPL and Labor demanded matches the downward sloping convex relation between wage and labor demanded.

% figure goes here

Same curves, but different interpretations:\\
In the first curves, we just have an exogenous specification of MPL\\

In the second curve, we have an optimal decision of labor defined as a function of exogenous wage.
\begin{eqnarray}
\frac{\partial Y(N)}{\partial N} &=& w\nonumber\\
\mbox{Define } G(N) &\equiv& \frac{\partial Y(N)}{\partial N}\nonumber\\
G(N^*(w)) &=& w\nonumber\\
\frac{\partial G}{\partial N} . \frac{\partial N^*}{\partial w} &=& 1\nonumber\\
\frac{\partial N^*}{\partial w} &=& \frac{1}{\frac{\partial G}{\partial N}} = \frac{1}{\frac{\partial^2 Y}{\partial N^2}}\nonumber\\
\mbox{More generally FOC : } Y ^\prime \left[ N^*(w)\right] &=& w\nonumber\\
\mbox{Define } Y^\prime(.) &\equiv& G(.)\nonumber\\
G\left[ N^*(w)\right] &=& w\nonumber\\
G\left[ N^*(w+\Delta w)\right] &=& w+\Delta w\nonumber\\
w+\Delta w &=& G\left[ N(w) + \Delta w N ^\prime(w) + \frac{\Delta w ^\prime}{2} N^{\prime\prime} (w)\right]\nonumber\\
&=& G\left[N(w)\right] + \left[ \Delta w N^\prime + \frac{\Delta w ^\prime}{2} N^{\prime \prime} \right] G ^\prime () + \frac{1}{2} \left[ \Delta w N^\prime\right]^2 G ^{\prime \prime}(.)\nonumber\\
\Delta w &=& \Delta w N^\prime G^\prime + \left( \frac{\Delta w^2}{2}\right) \left\{ N^{\prime \prime} G ^ \prime + (N^\prime)^2 G ^{\prime\prime}\right\}\nonumber\\
\Delta w^1 : 1 = N^\prime G ^ \prime \Longrightarrow N^\prime &=& \frac{1}{G^\prime} \\
\Delta w^2 : 0 = N^{\prime \prime} G ^ \prime +(N^\prime)^2 G^{\prime \prime}\Longrightarrow N^{\prime\prime} &=& \frac{-1}{G^\prime} (N^\prime) G^{\prime\prime} \nonumber\\
N^{\prime\prime} &=& \frac{-G ^{\prime \prime}}{(G^\prime)^3}\\
\mbox{Example : } Y &=& N^\alpha \nonumber\\
Y^\prime &=& G = \alpha N^{\alpha -1}\nonumber\\
G^\prime &=& \alpha (\alpha -1) N^{\alpha -2}\nonumber\\
G^{\prime\prime} &=& \alpha (\alpha -1)(\alpha -2) N^{\alpha -3}\nonumber\\
\mbox{Optimization : } \alpha N^{\alpha -1} &=& w\nonumber\\
N(w) &=& \left( \frac{w}{\alpha}\right) ^{\frac{1}{\alpha -1}}\nonumber\\
N^\prime(w) &=& \left(\frac{1}{\alpha -1}\right) \left(\frac{w}{\alpha}\right) ^{\frac{2-\alpha}{\alpha -1}} \left(\frac{1}{\alpha} \right)\nonumber\\
N^{\prime\prime}(w) &=& \left(\frac{1}{\alpha(\alpha -1)}\right)^2 (2-\alpha) \left(\frac{w}{\alpha}\right) ^{\frac{3-2\alpha}{\alpha -1}}\nonumber
\end{eqnarray}
\end{document}