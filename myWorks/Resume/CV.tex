\documentstyle{article}
\oddsidemargin=.15in
\evensidemargin=.15in
\textwidth=6in
\topmargin=-.5in
\textheight=9in
\parindent=0in

\pagestyle{empty}

\begin{document}

\begin{center}
{\Large Laura Anne Taalman} \\[.5pc]
Duke University Graduate Department of Mathematics \\
(919) 660-2829 $\;$ taal\verb|@|math.duke.edu $\;$ (919) 220-1359 \\[5pc]
\end{center}

{\large \bf Education} \\*[-.8pc]
\underline{\hspace{6in}} \\
\\
{\bf Ph.D. in Mathematics, Duke University, 1999} \\
Anticipate completion and defense of dissertation in May. \\
\\
{\bf Master of Science in Mathematics, Duke University, 1996} \\
\\
{\bf Bachelor of Science in Mathematics, University of Chicago, 1994}\\
\\
\\
{\large \bf Skills} \\*[-.8pc]
\underline{\hspace{6in}} \\
\\
{\bf Problem Solving} \\
Excellent analytical and logical reasoning skills.  Able to multi-task. 
Can learn new skills quickly. Able to lead or work within a group environment. \\
\\
{\bf Computer Languages} \\
Some experience with C++, HTML, Java, LISP, Pascal. Can become 
proficient in any of these (or other) languages upon request. \\
\\
{\bf Other} \\
Creative, motivated, innovative, stunningly beautiful. 
\LaTeX, mathematical ability (obviously).  Teaching skills. \\
\\
\\
{\large \bf Employment, Duke University} \\*[-.8pc]
\underline{\hspace{6in}} \\
\\
{\bf Instructor of Mathematics} \hfill {\it Spring 1996 to present\/} \\
Preparation and presentation of lectures, supervision of group work, 
writing and grading tests and quizzes, grading projects and papers, 
preparing and grading the final exam for the course. For some of the courses 
it was also necessary for me to create my own syllabus, 
as well as choose (or create) all of the the homework problems, worksheets, 
and labs. Math 25L and 26L are laboratory (``reform'') calculus courses; 
Math 19 and 104 are standard (``traditional'') courses. \\*[-2pc]

\begin{table}[h]
\begin{center}
\begin{tabular}{lll}
Math 19  & Precalculus             & Summer 1996 \\
Math 25L & Calculus-Precalculus I  & Fall 1996, Fall 1997, Fall 1998 \\
Math 26L & Calculus-Precalculus II & Spring 1996, Spring 1997, Spring 1999 \\
Math 104 & Linear Algebra          & Summer 1997 \\[-1pc]
\end{tabular}
\end{center}
\end{table}

{\bf Course Development} \hfill {\it Summer 1997, Summer 1998\/} \\
For the past two years I have spent my summers designing and developing
Math 25L and 26L.  This includes writing new syllabi and developing
course materials, policies, homework assignments, labs, and worksheets.
Each year my work (and the work of Jack Bookman)
has resulted in a sizable ``Coursepack'' used by all sections 
(for 25L in 1997, and 25L-26L in 1998). \\
\\
{\bf GRE Instructor} \hfill {\it Spring 1998\/} \\
Taught two three-week short courses in GRE preparation to a diverse
population of students. \\
\\
{\bf Lab Instructor} \hfill {\it 1994-5, Fall 1995\/} \\
Taught and supervised weekly Calculus ``Labs'' (calculator applications 
of the class material from that week) for 31L, 32L, and 25L; jobs included 
short lectures, helping the students with the material,
calculator ``troubleshooting'', and grading quizzes, 
projects, and reports. \\
\\
{\bf Calculus Tutor} \hfill {\it Fall 1994 to present\/} \\
Staffed weekly university ``Help Room'' where students go for help
in the undergraduate Calculus courses (25L-26L and 31L-32L). \\
\\
\\
{\large \bf Employment, University of Chicago} \\*[-.8pc]
\underline{\hspace{6in}} \\
\\
{\bf Teaching Assistant} \hfill {\it 1991-4\/} \\
Ran discussion/problem/review sessions twice a week that augmented the 
basic first-year Calculus course, graded papers and quizzes, some guest 
lecturing. \\
\\
{\bf Counselor, Young Scholars Program (YSP)} \hfill {\it Summer 1994 and 1995\/} \\
Gave lecture/discussion sessions and ran the computer lab. 
YSP presents talented high school students
with upper college-level mathematics and computer science material, 
namely finite fields and chaos theory in 1994, and knot theory and neural 
networks in 1995. \\
\\
{\bf Organizer, SESAME Conference} \hfill {\it Summer 1994\/} \\
``SESAME'' is a program for elementary and high school teachers; 
participants in the program attend a series of lectures and computer 
labs.  My job was both to help organize and coordinate the program, and to 
run computer labs in Mathematica and Geometer's Sketchpad. \\
\\
{\bf Research Assistant} \hfill {\it 1993-4\/} \\
Worked with Professor Paul Sally; constructed models of
polygons suitable for tessellation. \\
\\
{\bf College Core Tutor} \hfill {\it 1993-4\/} \\
Part of university tutoring staff that provided  
assistance to students in the core mathematics classes. \\
\\
\\
{\large \bf Other Employment} \\*[-.8pc]
\underline{\hspace{6in}} \\
\\
{\bf LaTeX Typesetting} \hfill {\it 1996 to present\/} \\
Conversion of mathematical or scientific papers into photo-ready
documents.  Jobs have included work
for Triangle Universities Nuclear Laboratory and papers in 
Computational Geometry. \\
\\
{\bf Private Mathematics Tutor} \hfill {\it 1990 to present\/} \\
Private mathematics tutor for various high school, undergraduate,
and graduate math courses. \\
\\
\\
{\large \bf Conferences} \\*[-.8pc]
\underline{\hspace{6in}} \\
\\
{\bf Park City Mathematics Institute} \hfill {\it Summer 1997\/} \\
Three week conference on Symplectic Geometry and Topology in Park City, Utah. \\
\\
{\bf Institute for Advanced Study} \hfill {\it Summer 1997\/} \\
Park City Women's Program in Symplectic Geometry and Topology in
Princeton, New Jersey.\\
\\
{\bf REU, University of Indiana} \hfill {\it Summer 1993\/} \\
Research Experience for Undergraduates; independent research (knot 
theory), lectures, thesis. \\
\\
\\
{\large \bf Awards and Honors} \\*[-.8pc]
\underline{\hspace{6in}} \\
\\
{\bf L.P. and Barbara Smith Award for Excellence in Teaching} \hfill {\it May 1998\/} \\
Sole recipient of the highest graduate student teaching award given
by the Mathematics Department. \\
\\
{\bf Dean's Award for Excellence in Teaching} \hfill {\it April 1997\/} \\
One of two recipients (university-wide) of this annual award from 
the Graduate School.  Teachers must be nominated by their students
(who write essays justifying the nomination) to be eligible for this
award. \\
\\
{\bf Departmental Teaching Award} \hfill {\it August 1997\/} \\
Annual award given by the Mathematics Department to graduate students
who demonstrate excellence in teaching. \\





\end{document}